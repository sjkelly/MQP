\documentclass[a4paper]{report}

\usepackage[english]{babel}
\usepackage[utf8x]{inputenc}
\usepackage{amsmath}
\usepackage{graphicx}
\usepackage{xcolor}
\usepackage{listings}
\usepackage{hyperref}
\usepackage{amssymb}
\usepackage[square,sort,comma,numbers]{natbib}
\usepackage[colorinlistoftodos]{todonotes}
\bibliographystyle{ieeetr}

\lstdefinestyle{julia}{
  basicstyle=\scriptsize\ttfamily,
  breaklines=false,
  backgroundcolor=\color{gray!10},
}
\lstset{style=julia,
        emph={julia},
        emphstyle={\color{green}\bfseries}}


\title{Computational Methods for Parametrization of Polytopes}
\author{Steve Kelly}

%toc settings
\setcounter{tocdepth}{4}
\setcounter{secnumdepth}{4}

\begin{document}
\maketitle

%put table of contents
\tableofcontents
\clearpage 


\begin{abstract}
The combinatorial and geometric realization of polytopes are outlined in
mathematical and computational terminology. With these two representations in
hand, various parametric forms may be constructed using vertex locations, edge
angles, and symbolic values. We have implemented
software which represents polytopes in a way useful for combinatorial
inspection and solid modelling using the Julia programming language.
These packages have been published to GitHub and are accessible to
mathematical researchers around the world through the Julia package manager.
\todo{ expand}
\end{abstract}


\chapter{Introduction}

Our objective is to develop a computational environment for the exploration
of parametric polytopes. A computational environment is one in which
we can apply rigourous defintional constraints on symbolic constructions, and
likewise manipulate them to reveal properties that may be of interest.
A parametric polytope is the union of two concepts. The first is the idea
of parameters, which are our unknown constraints in a system. A polytope
is a geometric object that has "flat" sides. The purpose of this paper
is to illucidate the prior work in computational representations of polytopes
and develop this work further.

\section{Motivation}


\section{Prior Work}

\todo{develop as body progresses}




\chapter{Mathematical Definitions}

We are interested in constructing parameteric polytopes for
mathematical exploration. An informal geometric
picture of a polytope has already been developed.
In this chapter we will develop a mathematical perspective of our problem
such that we can focus clearly on the computational aspects in the following
sections. 


\section{Spacial Terminology}

\subsection{Vector Spaces}

Given a field (in the algebraic sense) of numbers, a vector of dimensionality
N is formed by an N-tuple of numbers in the field. A vector space must
be closed under element-wise addition with another vector and element-wise
multiplication by a scalar value. Symbolically, given $s \in F$ where $F$ is a
field, and $X,Y \in V$ where V is the vector space over $F$ then
$X+Y \in V$ and $s*x \in V$ implies our closure property.
In this paper we will primarily use
the real numbers in euclidean N dimensional space, denoted as $\mathbb{R}^N$.
The term "point" will generally be used to describe a vector that may
described using numerical values.

\subsubsection{Closed Vector Spaces}

\todo{start with proper definition}
A "closed" space means a subset of something that contains all points in it's
boundary, including the boundary.
\todo{boundary is ambiguous}
 An "open" set conversely will not include
the boundary, but everything inside.
We sometimes use brackets to aid the representation of sets on a number line
for example: (1,2) for an open set and
[1,2] for a closed set. The quantity of the objects contained in our sets
will vary depending on the numeric domain we choose.
For example if we are using integers we have no
elements in the open set and 2 in the closed set. If it is in rationals we
have 1/2 included, and in the reals we have infinite points! What we have
constructed on the number line can also be thought of as open
and closed intervals.


\subsection{Solids}

Before we define a polyhedra, we must introduce a few notions. These are
solids and orientation. Solids have been studied since
antiquity and for our purposes we will define them as constructions in
three dimensional space with finite volume.
For example, a plane which partitions space is not a solid
since either partition is unbounded, however the intersection of planes
could form a solid.

Orientation is a parallel concept which allows us to specify how geometric
objects contain space. As an example, let us go back to our partition of
space with a plane. If we are on our way to construct a solid, it is
neccessary to choose the one part to keep and the other to discard. This is
the purpose of orientation. In the case of the plane, this follows from the
definition, $ax+by+cz+d=0$. More lucidly, lets look at the signed distance
of a point from the plane, computed as:
\begin{equation}
D = \frac{ax+by+cz+d}{\sqrt{a^2+b^2+c^2}}
\end{equation}
For simplicity, let's look at a plane parallel to the X and Y axes passing
through z = 1. Thus $a$ and $b$ are set to 0, $c$ set to 1, and
$d=-1$
Simplifying our formulation we have:
\begin{equation}
D = 0x+0y+z-1 = z-1
\end{equation}
At $z=1$ we see we are on the plane, however at $z=0$ and $z=2$ we get -1 and 1
respectively. The sign of the distance is our indicator of orientation. We can
choose an arbitrary convention as to which partition we will count, but akin
to the "right hand rule" in physics, the normals of the partition must point
outwards. In the case of the plane the normal is the vector $(a,b,c)$, which
in our realization is the upwards vector $(0,0,1)$. Since the convention is
such that the normals point outward, the partition we would consider in a
solid is all points \emph{in the opposite direction of the normal}.
Also to note is the importance of sign in our distance function. We can exploit
this behavior to indicate containment when performing set operations on
spatial partitions. Negative values thus indicate a point inside the partition
of interest. We have chosen this convention for this paper due to
it's ubiquity in computation frameworks. In the field of computer graphics
using descrete polytopes, this is often referred to as "winding order".

\section{Combinatorial Representation of a Polytope}

\subsection{Simplices}

A simplex (plural simplices) in N dimensional space is the minimal set of
descrete points whose convex hull form a closed subset of dimensionality N.
It is often thought
of as the generalization of a tetrahedra into N dimensions. In one dimensional
space, this is a closed interval or line segment. In two dimensions it is the
triangle, and in three it is the tetrahedra. These are called a 1-simplex,
2-simplex, and 3-simplex respectively. We will represent them as 

\todo{set definition}


\begin{figure}[h!]
  \centering
    \includegraphics[width=0.4\textwidth]{img/800px-1-simplex_t0.png}
    \includegraphics[width=0.4\textwidth]{img/800px-2-simplex_t0.png}
    \includegraphics[width=0.4\textwidth]{img/800px-3-simplex_t0.png}
    \includegraphics[width=0.4\textwidth]{img/800px-4-simplex_t0.png}
  \caption{Examples of Simplices}
  \label{fig:simplices}
\end{figure}

We notice that the 1-simplex is formed by two 0-simplices, a 2-simplex is formed
by three 1-simplices, and the 3-simplex by four 2-simplices. The components of
these compositions are called "faces". The 0-face is often called a vertex
and the 1-face an edge. If we tabled the quantities of each M-face in an
N-simplex out, they form
Pascal's triangle and thus follow the binomial coefficient.

\begin{equation}
{N+1}\choose{M+1}
\end{equation}

Simplices will be our most basic geometry we use for formulating the descrete
form of a polytope. In fact, the two- and three-simplices are already
polytopes!

\subsubsection{Orientation}

Since we will eventually like to construct a functional representation
of a polytope from a combinatorial form, we must consider orientation.
In our case, we will consider how the construction is ordered. More
specifically, how the points are specified. Let us consider the most basic
case on a number line of an interval or 1-simplex. If we choose 1 and 2 to
form our closed interval we may express this as [1,2] or [2,1]. In this
case the result really would not change to any practical effect, and more so
[2,1] may be considered an incorrect representation in some dicisplines.
This example simply shows
how ordering may be communicated symbolically.

In the case of the 2-simplex it becomes more clear. Given three unique points
$a$, $b$, and $c$ specifying a triangle we have two possible orderings. Namely
a "clockwise" and "counterclockwise" orientation. We see in effect the orientation
of the edges is the same in the forumulations $[a,b,c]$, $[c,a,b]$, and $[b,c,a]$,
since the edges $[a,b]$, $[b,c]$, and $[c,a]$ are all formed identically.
However the formulations $[c,b,a]$, $[a,c,b]$, and $[b,a,c]$ form the edges
$[b,a]$, $[c,b]$, and $[a,c]$. This is opposite to the first formulation we
specified!

\begin{figure}[h!]
  \centering
    \includegraphics[width=0.75\textwidth]{img/Winding_order.png}
  \caption{The two possible orientations of a triangle given 3 points.}
  \label{fig:tri_orientation}
\end{figure}


Thus we may define the orientation of an simplex as a class of orderings taken
by even permutation (cite Hempel Thesis). More so we see that the orientation
of the edges follow from the orientation of the simplex. This property
will be important when we analyize the structure of Polyhedra.



\todo{John M. Lee Introduction to Topological Manifolds}
\cite{Lee_2011}

We now have the tools neccesary to construct a combinatorial representation
of a polytope. In the functional representation we were interested in
extracting geometric information about points and their distances from the
polytope. The combinatorial representation will help us observe the
relationships between the faces. To construct an N-polytope we will use 
N-1 simplices. using set 

\todo{Simplicial Complexes}

\subsection{Construction of the Combinatorial Representation}


\section{Implicit Functional Representations of Solids}

\subsection{Hyperplanes}

In multiple dimensions this is more interesting since we may construct
various, rather arbitrary, geometries to make a closed space.
One common example is a hyperplane. A hyperplane is simply a generalization
of the a plane into arbitrary dimensions, with the property it is
of dimensionality N-1. For example if we are in 2D space, our hyperplane
is a line since it partitions our space into two parts. Likewise in 3D
space this is a plane. If we define a hyperplane functionally using vector
notation we can extract some interesting properties.
For simplicity in this example let us assume we have a hyperplane which
cuts through the origin.
If we let 
$\vec{x}$ be an arbitrary point in $N$ dimensional space,
and $\vec{a}$ be the slopes of each axis, then one functional construction is
simply the dot product, $dot(\vec{a},\vec{x})$. If x is on the hyperplane
the function will be equal to zero. If it is not zero,
then we may determine which side of the partition the point lies on.
Thus as a set representation the hyperplane is:

\begin{equation}
\{f(x)=dot(\vec{a},\vec{x})|f(x)=0,x \in \mathbb{R}\}
\end{equation}

\todo{define hyperplane sooner}
We notice that the hyperplane cuts space, but does not create a closed
subspace. What we are after is a "solid", which is the composition of
such objects forming a closed space. In fact, this leads to a proper definition
of a closed set in topology. We will say that \emph{a set is closed if and only if it
contains all of its limits}\cite{007054235X}. In the case of the lone hyperplane
it is not bounded, so itself therefore not closed (by not having limits). Thus
we must compose hyperplanes to form closed spaces!


\todo{Implicit Surface and Distance Field}

If we were to functionally define the boundary of a solid a predicate of the
form $f(x) = 0$ would suffice, where the solid is defined by all $x$.
However we previously mentioned that it is
useful in computation to define a function that returns infomation about
the membership of a point in the solid. Further more we can define our functional
solid to return a value corresponding to the shortest distance to the boundary.\cite{Olah_2011}
These functions are sometimes called "implicit" forms in solid modelling, but
more precisely they generate a distance field.

A sketch of this behavior in one dimension
can be seen in Figure \ref{fig:implicit-sketch}.

\begin{figure}[h!]
  \centering
    \includegraphics[width=0.75\textwidth]{img/implicit_sketch.png}
  \caption{Number line illustrating the construction of an implicit signed function}
  \label{fig:implicit-sketch}
\end{figure}

\subsubsection{Affine Transforms}

Functional representations can naturally deal with affine
transforms\cite{Henderson_2002}. Given some transform associated with a
solid, one simply applies the inverse transform to check membership.
The key word here is "associated" since for our purposes we will define
geometries without consideration of their tranforms. This is one aspect
in which computation will aid us extensively. If we let $f$ be a functional
representation of a solid or spatial partition
which generates a signed distance field, $A$ be the transform of the object,
and $x$ be the point whose membership in the solid is in question. It then
follows that the distance field formed by $f$ transformed by $A$ is obtained
by $f(A^{-1}*x)$.

\begin{figure}[h!]
  \centering
    \includegraphics[width=0.5\textwidth]{img/Fractal_fern_explained.png}
  \caption{Number line illustrating the construction of an implicit signed function}
  \label{fig:implicit-sketch}
\end{figure}

\todo{Affines are groups}

\subsubsection{Orientation}


\subsection{Set operations on Distance Fields}


One can also compose distance fields with logical operations. 
Below are basic set operations defined for these functions using our
sign conventions:

\begin{equation*}
\cap : \mathtt{min}(f_1,f_2) \\
\end{equation*}
\begin{equation*}
\cup : \mathtt{max}(f_1,f_2) \\
\end{equation*}
\begin{equation*}
\neg : -\mathtt{f}_1
\end{equation*}

\todo{proof/cite}

It follows that the ``difference"
of $f_1$ and $f_2$ is the intersection of $f_1$ with the negation of $f_2$,
$\mathtt{max}(f_1,-f_2)$. Thus we may compose functional representations
of geometry using the language of set operations, namely, union, difference,
intersection, and negation.
The mathematical analyst might have trouble with these formulations because
such operations create discontinuities. To illucidate this problem we may
use the language of norms. A distance field as we have defined it behaves
like the inf-norm, or maximum norm. If we choose a point and take a projection
to the surface which forms the shortest path, our distance value
will be linear. This is not very useful for any kind of differential
analysis since we only have first order derivatives.
It may be useful for a parametric mathematical construction of a polytope
to have differentiable properties. Rvachev functions
provide one solution to this problem.


\subsubsection{Rvachev Functions}

In the 1960's Vladimir Rvachev produced a method for handling the "inverse
problem of analytic geometry". His theory consists of functions which provide a
link between logical and set operations in geometric modeling and analytic
geometry.\cite{shapiro1991theory} While attempting to solve boundary value problems,
Rvachev formulated an equation of a square as
\begin{equation*}
a^2 + b^2 − x^2 − y^2 + \sqrt[]{( a^2 − x^2 )^2 +( b^2 − y^2 )^2} =0
\end{equation*}

Implicitly, the sides of a square can be defined as $x= +/- a$ and $y= +/- b$.
The union of these two is a square. By reducing the formulation of the square
he generalized an expression for the union between two functions.
\begin{equation*}
\cup : f_1 + f_2 + \sqrt[]{f_1^2 +f_2^2} = 0
\end{equation*}

Likewise we can see that intersections and negations can be formed for logical
completion.
\begin{equation*}
\cap : f_1 + f_2 - \sqrt[]{f_1^2 +f_2^2} =0 \\
\end{equation*}
\begin{equation*}
\neg : -f_1
\end{equation*}

These formulations can be modified for $C^m$ continuity for any $m$.

\todo{clean}

The Rvachev construction is of hypothetical interest in the presentation,
but shows how geometric
constructions can converge with analytic constructions.

In 2000 Rvachev published an english-language proof of the completeness
\cite{Rvachev_Sheiko_Shapiro_Tsukanov_2000}

\cite{shapiro2007semi} In addition Pasko, et. al. have shown that Rvachev
functions can serve to replace a geometry kernel by creating logical
predicates. \cite{pasko1995function} Their research also establishes the
grounds for user interfaces and environment description. For this work a
practical implementation will most likely leverage their insights.
Rvachev and Shapiro have also shown that using the POLE-PLAST and SAGE
systems a user can generate complex semi-analytic geometry
as well.\cite{rvachev2000completeness} 

One of the most general expositions in the English language of R-Functions
applied to BVPs is
Vadim Shapiro's``Semi-Analytic Geometry with R-Functions". \cite{shapiro2007semi}
Unfortunately, no monographs about R-Functions exist in the English literature.
Most literature is in Russian, however many articles presenting applied
problems using the R-Function Method. \cite{voron2010}

Such a system for analytic geometry can be developed further. In the context
of an Eulerian flow field, a distance field over a function that
generates partial derivatives could be a fast numerical computation method.

\todo{clean}



\subsection{Construction of the Implicit Functional Representation}

We now have the mathematical concepts needed to define a polytope using the
language of hyperplanes. Using vector notation we may define a hyperplane 
as an implicit function generating a distance field by:

\begin{equation}
f(x) = \frac{a \cdot x }{\sqrt{a \cdot a}}
\end{equation}

where $a$ are the slopes and $x$ being the point in question. This function
will be zero when $x$ lies on the plane.

\cite{Polyhedra}



\chapter{Computational Definitions and Grammar}

Programming languages are the grammar and syntax a computer presents to a user.
This project is fundamentally exploratory in nature and seeks to generate
understanding of geometric relationships using the intersection of
mathematical and computational rigor. We have chosen to use the Julia
programming language due to comfort of development, and an abundance of
supporting libraries for mathematical computation. In this chapter
we will give a brief introduction to many computing concepts and illustrate
how Julia advances them to meet our needs well.

\section{History}
Julia is a programming language first released in early 2012 by a group of
developers from MIT. The language targets technical computing by providing a
dynamic type system with near-native code performance. This is accomplished by
using three concepts: a Just-In-Time (JIT) compiler to target the LLVM framework,
a multiple dispatch system, and code specialization\cite{bezanson2012julia}
\cite{Bezanson_Edelman_Karpinski_Shah_2014}.
More simply, the language is designed to be dynamic in a way that allows
rapid prototyping of code and understandable to a reader, yet provides
a design amicable to performance optimizations and specialization.
Dynamic type systems allow the programmer to ignore or selectively
specify type information, such
as the bytes in an integer, and alow the compiler to infer this information
based on the input types. JIT compilation means code is compiled during runtime
which allows functions to be recompiled and thus optimized for varous
data types.
The syntactical style is similar to MATLAB and Python.
The language implementation and many libraries are available under the
permissive MIT license.\footnote{\url{http://opensource.org/licenses/MIT}}

Benchmarks have shown the language can consistently perform within a factor of
two of native C and FORTRAN code.\footnote{\url{http://julialang.org/benchmarks}}
This is enticing for a solid modeling application and for numerical analysis,
as the code abstraction can grow organically without performance penalty.
In fact, the authors of Julia call this balance a solution to the 
``two language problem". The problem is encountered when abstraction in a
high-level language will disproportionately affect performance unless
implemented in a low-level language. In the next sections we will compare
the expressibility and performance to other languages.

\section{Comparisons}

Many languages are as fast as Julia but sacrifice expressibility.
In Figure \ref{fig:juliabench} we can see some comparisons to other programming
languages. This was developed by the Julia core team, and illustrates that
Julia is highly competitive in performance. Again, these results stem from
the compiler and language design. In Figure \ref{fig:juliaexpr} we can see
these results normalized against code length. The Julia code is quite short,
yet consistently achieves good performance.
Thus the programmer may write less code and spend less time waiting
for results in an interactive environment, which makes Julia a great choice
for exploratory programming.
Much of this comes down to the innovated type and function system.\cite{Chen2014} We will
discuss these more in depth later.

\begin{figure}[h!]
  \centering
    \includegraphics[width=1.0\textwidth]{img/juliabench.pdf}
  \caption{A comparison of programming languages and performance.}
  \label{fig:juliabench}
\end{figure}

\begin{figure}[h!]
  \centering
    \includegraphics[width=1.0\textwidth]{img/expressability.pdf}
  \caption{The results in Figure \ref{fig:juliabench} normalized for code length. (Courtesy of Simon Danish)}
  \label{fig:juliaexpr}
\end{figure}


In 1972 Alan Kay introduced the terms
``class" and ``object" to describe a coupling of data and functionality.\footnote{\url{http://gagne.homedns.org/~tgagne/contrib/EarlyHistoryST.html}}
An object is an instance of a class, which contains the definitions of 
functions and member data. Computer Scientists
call this "Object Oriented Programming" (OOP).
Languages such as C++, Java, and Python all subscribe to this paradigm.
In Python this looks like the following:
\begin{lstlisting}
class Foo:
    foo1
    foo2
    def add_to_foo1(self, x):
        self.foo1 += x
\end{lstlisting}

This system positively enables specialization of functionality, but due
to the coupling of data with functions it becomes a challenge to extend
functionality. Languages for scientific computing generally avoid the
``traditional" notions
of OOP, preferring rather to seperate data from
functionality. In Table \ref{tab:types} we can see a
comparison of type systems used in scientific computing languages. Here ``Type
system" can be either dynamic or static, where in a static system the programmer
needs to specify to the program compiler how data is transformed in a function.
Generic functions allow a single function name, for example \texttt{sum}, to have
multiple defintions with execution contigent upon the matching of argument
types. Many programmers may first encounter generic functions through the
term ``overloaded" function as well.
The definition of a parametric type is more nuanced, but generally
means that the definition of a type may vary based on the types of it's
member data. We will dedicate a section to the explaination of type parameters.
In the next
few sections these ideas will hopefully be clarified and the implications of
multiple dispatch and the relation to OOP will be developed further.


\begin{figure}[h!]
  \centering
    \caption{A comparison of functions, typing, and dispatch.}
    \begin{tabular}{ l | l l l}
    Language & Type system & Generic functions & Parametric types \\
    \hline
    Julia & dynamic & default & yes \\
    Common Lisp & dynamic & opt-in & yes (but no dispatch) \\
    Dylan & dynamic & default & partial (no dispatch) \\
    Fortress & static & default & yes \\
    \end{tabular}
  \label{tab:types}
\end{figure}


\section{Functions}
Julia is an experiment in language design. Much of the advancement
revolves around the representation of data and the execution of functions.
The language is optionally or dynamically typed, which means function specialization on types
is inferred by the compiler without user intervention. This is an idea
first utilized in the Hadley Milner's ``ML" which was created to develop theorem
provers\cite{Harper_1997}. The compiler analizes program flow and is able
to infer the types of variables and function returns.
A basic example of inference in Julia is shown below:
\begin{lstlisting}
julia> increment(x) = x + 1
increment (generic function with 1 method)

julia> increment(1)
2

julia> increment(1.0)
2.0
\end{lstlisting}\footnote{The REPL (Read-Eval-Print-Loop) allows interactive
evaluation of Julia code. It is highly useful for exploration and testing of
ideas in the language.
Blocks starting with "\texttt{julia>}" represent input and the preceding
line represents output of the evaluated line.
}
The \texttt{increment} function was defined for any \texttt{x} value. When the
\texttt{1}, an
integer type was passed as an argument, an integer was returned. Likewise
when a floating point, \texttt{1.0} was passed, the floating point
\texttt{2.0} was returned.

Let's see what happens when we try a string:
\begin{lstlisting}
julia> increment("a")
ERROR: MethodError: `+` has no method matching +(::ASCIIString, ::Int64)
Closest candidates are:
  +(::Any, ::Any, ::Any, ::Any...)
  +(::Int64, ::Int64)
  +(::Complex{Bool}, ::Real)
  ...
 in increment at none:1
\end{lstlisting}

The problem is that the \texttt{+} function is not implemented between the
\texttt{ASCIIString} and \texttt{Int64} types.
We need to either implement a \texttt{+} function
which might be ambiguous, or specialize the function for \texttt{ASCIIString}.
A specific implementation is preferrable in this case:
\begin{lstlisting}
julia> function increment(x::ASCIIString)
           ASCIIString([increment(c) for c in x])
       end
increment (generic function with 2 methods)
\end{lstlisting}
The line \texttt{x::ASCIIString} is called a ``type annotation" and
states that \texttt{x} must be a subtype
of \texttt{ASCIIString}. This allows one to control dispatch of types to
functions,
since Julia will default to the \emph{most specific implementation}
for the type.
Since\texttt{ASCIIString} is a series of 8 bit characters, we can iterate over the
string and increment each character individually. The \texttt{[]} indicates we are
constructing an array of characters to pass to be passed to the \texttt{ASCIIString}
type constructor. Now we see our example works:
\begin{lstlisting}
julia> increment("abc")
"bcd"
\end{lstlisting}

What was demonstrated here is the concepts of specialization and multiple
dispatch, both are highly coupled topics.
Each function call in Julia is specialized for types if possible.
This means the author only has to write a few sufficently abstract
implementations of functions. If special cases occur, multiple functions
with different arity or type signatures can be implmented. Explicitly
this is called multiple dispatch. In practice by the user this looks like
abstracted or generic code if done well so many types can be handled
by one function.
To the computer, this means choosing or generating the most specific and
performant method\footnote{Functions and methods are distinct in Julia. A
function may be thought of in the mathematical sense. A method is a
function specialized on types with unique machine code.}
Let's go back to the integer and floating point
example. Below is the LLVM assembly generated for each method:
\begin{lstlisting}
julia> @code_llvm increment(1)

define i64 @julia_increment_21458(i64) { // <return type> <function name>(<arg type>)
top:
  %1 = add i64 %0, 1
  ret i64 %1 // return <return type> <return id>
}

julia> @code_llvm increment(1.0)

define double @julia_increment_21466(double) {
top:
  %1 = fadd double %0, 1.000000e+00
  ret double %1
}
\end{lstlisting}

Note we have annotated the LLVM code so this is understandable. 
The only real similarity is the line count. Each one of these functions are generated by the
Julia compiler at run time.

Many of the concepts used for performance also serve as methods for
expressability. In this case, multiple dispatch used by the compiler for
specialization of functions reveals itself as a way for the user to
specialize over many types. In summary, the
basic steps in generating native computer code from a function are to:
\begin{enumerate}
\item Parse the expression
\item Infer type information
\item Generate native machine code and optimizations
\end{enumerate}

\section{Types}

\subsection{Mutability and Data Packing}
Types and immutables are containers of data. The primary difference between
the two is the notion of ``mutability". Types are mutabile, immutables are 
immutable. What does this mean? Let's break something first via the REPL:
\begin{lstlisting}
julia> type FooIsMutable
           a
       end

julia> f = FooIsMutable(1)
FooIsMutable(1)

julia> f.a
1

julia> f.a = 2
2

julia> f.a
2

julia> immutable FooIsImmutable
           a
       end

julia> f = FooIsImmutable(1)
FooIsImmutable(1)

julia> f.a
1

julia> f.a = 2
ERROR: type FooIsImmutable is immutable
\end{lstlisting}

What just happened demonstrates the contract defined by mutability. Mutable
objects, which is an instance of a type (i.e. \texttt{f}), can have their fields
(i.e. \texttt{a}) changed. Immutables cannot. The immutable contract helps develop
a notion of functional purity. To the user this means immutables are defined
by their values. This can be of great benefit to avoid errors and establish
concrete equality between types, such as vectors.
Practically this can be of great benefit to
the compiler to determine invariants and eliminate pointers in a datatype.
For example:
\begin{lstlisting}
julia> a = (1,2,3)
(1,2,3)

julia> b = typeof(a)
Tuple{Int64,Int64,Int64}

julia> isbits(b)
true

julia> a = ([1],[2],[3])
([1],[2],[3])

julia> b = typeof(a)
Tuple{Array{Int64,1},Array{Int64,1},Array{Int64,1}}

julia> isbits(b)
false
\end{lstlisting}

\texttt{isbits} ask the question ``will this type be tightly packed in memory
without pointers to values"? A
\texttt{Tuple} is a fixed-length set of linear, ordered, data. It has syntax for
construction with \texttt{()}. In computations we want our data be close together
for fast access. In modern times we call such data ``cache friendly", or
``cache localized", which means the computer may store the data in
registers closer to the CPU.
Immutability helps us achieve this. Let's look that the
types inside the 3-tuples and see their \texttt{isbits} status:
\begin{lstlisting}
julia> isbits(Array{Int64,1})
false

julia> isbits(Int64)
true
\end{lstlisting}
Why is this the case? We see that \texttt{Int64} is bits, because it is literally
64 bits. In Julia a \texttt{bitstype} behaves similar to an immutable, and is identified
by value. For example the defintion for \texttt{Int64} is \texttt{bitstype 8 Int64}
which means an \texttt{Int64} is 8 bytes long.
\texttt{Array\{Int64,1\}} is a mutable data type that can vary in size.
This means
the \texttt{Tuple} needs to store the arrays as references to memory,
in this case a
pointer. When iterating over a data set, such a ``pointer dereferences" (this is
jargon for accessing the data in memory pointed to by a pointer), can be costly.
Modern CPUs excel when data is linearly packed and pointer-free. The
data can be brought into the CPU's memory cache and registers only once
and computed without
shuffling between cache and RAM. The cost of lookup time between the cache and
RAM generally differs by several orders of magnitude.

\subsection{Parameters}



\section{Macros and Generated Functions}
Julia is a descendant of the Lisp family of programming languages. Lisp
is a portmanteau for ``List Processing". The language was designed to address
the new notion of ``types", specifically in application to Artificial
Intelligence (AI) problems\cite{McCarthy_1966}. The notion of an ``S-Expression"
was introduced in McCarthy's seminal work, ``Recursive functions of symbolic expressions and their
computation by machine". These statements use parenthesis
to denote functions and arguments. Below is an an example of S-Expressions
for addition and multiplication.

\begin{lstlisting}
> (+ 1 1)
2

> (* 3 4)
12
\end{lstlisting}

This syntax is noted for it's mathematical purity.
However it can be a syntactic difficulty for many.
Most of the current popular programming languages
use variants of ALGOL syntax, which is noted for being more readable
\cite{Hoare}.
Julia also uses ALGOL syntax, but is converted to S-Expressions after parsing
\cite{julia_internals}.
This enables
many of the mathematically pure relations we seek to achieve.
In addition S-Expressions are highly conducive to source transforms.
This develops a notion of ``Homoiconicity", where the representation of
program structure is similar to the syntax. In Julia we use this property
to make ``macros" which enable source code to be transformed based on
the syntactical structure before compilation.

Generated functions perform a similar function as macros, but at the function
level. They enable source code to be procedurally generated based on types.
This allows the user fine-tuned control of the compilation process, and
will allow optimizations to be performed that are not currently
available in the compiler.
Surveys of Computer Science literature show that such a concept is new in
a programming language that uses type inference
\cite{julia_metaprog}.
However the use of generated functions is generally frowned upon by the
Julia community since it makes compilation more difficult since type inference
has to be run multiple times before compilation may happen.
This often slows down trivial functions by
several orders of magnitude, and should be only used if a method is called
many times and performancde is critical.

We will omit an introduction of macros and generated functions
as they are advanced langauge
features. For our purposes a basic understanding of the terminology will
be sufficient.


\section{Numerical Robustness}

Numerical robustness is a perennial problem in computational geometry\cite{Shamos_1999}.
Multiple
approaches exists for various numeric types. Floating points are by far
the most difficult to deal with. Tools such as Gappa have been developed so
algorithm writers can check their invariants when using floating points\cite{Gappa}.
Such tools complicate software development and are not an accessible option
for the casual researcher.

One of the most common problems formulated is to determine whether or not a
point is collinear with a line segment. Shewchuk has one of the most pragmatic
and robust treatments on this topic\cite{Shewchuk}. Kettner, et. al. have also
developed more examples where numerical robustness is critical\cite{Kettner_Mehlhorn_Pion_Schirra_Yap_2008}.

Julia's GeometricalPredicates package \footnote{\url{https://github.com/JuliaGeometry/GeometricalPredicates.jl}}
uses the approach outlined by Volker Springel, which requires all floating point
numbers to be scales between 1 and 2\cite{Springel_2010}. This has the downside
of significantly reducing the available resolution to 50\% of the available
floating point numbers.

A simpler, although less applicable, approach is to work
within integer space. Developing a system around this is of interest. For
example, it should be possible to specify a minimum unit (e.g. microns)
and perform all computations in integer space assuming this does not exceed
the needed resolution. More importantly, modern CPUs have integrated 128 bit
Integer support. 170141183460469231731687303715884105727 is a lot of microns.





\chapter{Implementation}

In this section we will begin to outline the implementation of
various forms of parametric polytopes.

\section{Survey of Available Packages}

In chapter 3 we outlined the rationale for using Julia for mathematical
computer programming. An additional impetus was the familiarity of the
geometry packages. There will be various references to these and they
are outlined below so the reader may become familiar with the utilities
available.

\subsection{GeometryTypes.jl}

GeometryTypes.jl provides datatypes and basic operations for computational
geometry. This package began as a unification of types located in
HyperRectangles.jl, Meshes.jl, and GLAbstraction.jl. The initial
types were polygonal meshes and bounding boxes, but now encompasses
datatypes for solid modeling, data visualization, and
geographic information systems.
With the introduction
of this package the community made some initial progress on designing
types that can be used for computation on the CPU and GPU, however
GPU targets are rapidly evolving and the focus has shifted from geometric
operations to array operations. Much of
our basic combinatorial analysis operations and data types have be contributed
to this package.

\url{https://github.com/JuliaGeometry/GeometryTypes.jl}

\subsection{FileIO.jl and MeshIO.jl}

FileIO.jl is a package that unifies various file loaders that existed in
the Julia package ecosystem under one import. The purpose is to
allow users to simply call the \texttt{save} and \texttt{load}
functions with file
information inferred from file extensions, magic numbers, or data types.
MeshIO.jl is one such packge that provides file loaders for
polygonal mesh data. The file formats supported as of this writing include
obj, stl, ply, off, and 2dm. This package may be useful for importing
polytope data from other programs such as Blender or AutoCAD, or generating
large data sets.

It should be noted that Julia has a \texttt{serialize} function, which will
save a datatype in full fidelity and in compact binary. Since Julia is
yet to reach a 1.0 release, this function is considered unstable. Once
\texttt{serialize} is stable, it will be the preferred method of saving
data sets to the computers storage drive.

\url{https://github.com/JuliaIO/FileIO.jl}

\url{https://github.com/JuliaIO/MeshIO.jl}

\subsection{Meshing.jl}

Meshing.jl provides algorithims for converting signed distance field
data into polytopes. The two algorithms currently provided are
the Marching Cubes (MC) and Marching Tetrahedra (MT) algorithms. For this project
we added the Marching Cubes algorithm which is twice as fast as the
Marching Tetrahedra algorithm. The import difference between the two is
performance and manifold mesh generation. The MT algorithm generates manifold
meshes, but generates more faces (costing memory) and is slower.
It is useful for generating
meshes from noisy data or applications where manifold meshes are required
such as finite element analysis and 3D printing. The Marching Cubes algorithm
is less costly for the computers resources, and is helpful for visualization
applications where user experience is important.

\url{https://github.com/JuliaGeometry/Meshing.jl}

\subsection{Meshes.jl}

Meshes.jl is currently a meta-package\footnote{Meta-package means
little or no code is contained in the package besides imported code
from other packages. It is
often used for version stability or usability purposes.} that imports elements on GeometryTypes.jl,
FileIO.jl, and Meshing.jl. It is one of the older packages in the
Julia package ecosystem and was an early center of collaboration before
the scopes began to expand. Releases before Meshes.jl became a meta-package 
are maintained for insitutional users. The name space is held to
allow for a center for experimentation as stability in the base packages
becomes more neccesary.

\url{https://github.com/JuliaGeometry/Meshes.jl}

\subsection{ParametricPolyhedra.jl}

ParametricPolyhedra.jl is a package used for solving constraints on
triangular faces of a polyhedra. The intention of this package
is to allow polyhedra to be specified via angles and edge lengths.
It draws heavily from the resources available in GeometryTypes. Since
it uses algorithms to define the types and is some what domain
specific at this point, we opted to make it a seperate package.

\url{https://github.com/sjkelly/ParametricPolyhedra.jl}

\subsection{GeometricalPredicates.jl}

GeometricalPredicates.jl is a package that provides numerically
robust primitives and algorithms for computing incircle, circumcircle, and
intriangle calculations. The approach to numerical robustness is used by
the Illustric Simulation, and outlines in Volker Springel's paper
"Galiliean-invariant cosmological hydrodynamical simulations on a moving mesh"\cite{Springel_2010}.
The essence of the approach is to restrict values in 64 bit floating points
between 1 and 2 since the exponent component is constant. This allows
128 bit integers to be used for overflow calculations.

\url{https://github.com/JuliaGeometry/GeometricalPredicates.jl}

\section{GeometryTypes.jl Implementations}

\section{Simplex}

We began by implementing a Simplex type in GeometryTypes.jl,
defined as follows:

\begin{lstlisting}
"""
A `Simplex` is a generalization of an N-dimensional tetrahedra and can be thought
of as a minimal convex set containing the specified points.

* A 0-simplex is a point.
* A 1-simplex is a line segment.
* A 2-simplex is a triangle.
* A 3-simplex is a tetrahedron.

Note that this datatype is offset by one compared to the traditional
mathematical terminology. So a one-simplex is represented as `Simplex{2,T}`.
This is for a simpler implementation.

It applies to infinite dimensions. The sturucture of this type is designed
to allow embedding in higher-order spaces by parameterizing on `T`.
"""
immutable Simplex{N,T} <: AbstractSimplex{N,T}
    _::NTuple{N,T}
end
\end{lstlisting}

With the definition in GeometryTypes, we afford ourselves two notions of
dimensionality. Our first parameter \texttt{N} gives us the total dimensionality
of the simplex. We will notice that our convention is offset by positive one
compared to the mathematical terminology. This is due to Julia not allowing
arithmetic in type defintions. There are a few approaches to circumvent this
issue, but they either make the datatype larger or sacrifice strong
type inference.

The second parameter, \texttt{T} is the type of the points. We will see that
point may be symbolic in nature, or have their own dimensionality
expressed independent of \texttt{N}.
For example in Julia we
may
prefix a colon to an identifier and make it a symbolic value which is reflected
in the type information:

\begin{lstlisting}
julia> using GeometryTypes

julia> Simplex(:x,:y,:z)
GeometryTypes.Simplex{3,Symbol}((:x,:y,:z))
\end{lstlisting}

In this example we have created a 2-simplex with symbols \texttt{:x, :y, :z}.
\texttt{N} is 3, and \texttt{T} has become \texttt{Symbol}.
Symbolic representation will allow us to create simple combinatorial
analysis.
Likewise we can construct concrete types:

\begin{lstlisting}
julia> Simplex(Point(0,0,0), Point(1,1,1))
GeometryTypes.Simplex{2,FixedSizeArrays.Point{3,Int64}}((FixedSizeArrays.Point{3,Int64}((0,0,0)),FixedSizeArrays.Point{3,Int64}((1,1,1))))
\end{lstlisting}

This last example illustrates how \texttt{N} and \texttt{T} may give us
two notions of dimensionality in the Simplex.
Here we have constructed a line segment in 3D space. The Simplex is of
size two but the space it occupies is three dimensional. This way it acts
similar to a fixed size vector, but the type implies all points are on the
convex hull. Unfortunately it may also be possible to construct a Simplex
using points of dimension less than that of the Simplex, which would
not hold to our contract of linear independence.
More so we may also decompose its

Below is an example of a high performance implementation of Simplex decomosition:

\todo{update code example}


\section{HomogenousMesh Type}

Prior to this project, GeometryTypes primarily provides for Polygonal Mesh
type that is well tuned for operations on the CPU and GPU. It is defined
as follows:

\begin{lstlisting}
"""
The `HomogenousMesh` type describes a polygonal mesh that is useful for
computation on the CPU or on the GPU.
All vectors must have the same length or must be empty, besides the face vector
Type can be void or a value, this way we can create many combinations from this
one mesh type.
This is not perfect, but helps to reduce a type explosion (imagine defining
every attribute combination as a new type).
"""
immutable HomogenousMesh{VertT, FaceT, NormalT, TexCoordT, ColorT, AttribT, AttribIDT} <: AbstractMesh{VertT, FaceT}
    vertices            ::Vector{VertT}
    faces               ::Vector{FaceT}
    normals             ::Vector{NormalT}
    texturecoordinates  ::Vector{TexCoordT}
    color               ::ColorT
    attributes          ::AttribT
    attribute_id        ::Vector{AttribIDT}
end
\end{lstlisting}

The first thing to note is the provisions for attributes, colors, and textures.
These are used for mapping textures and/or colors to polygons via visualization
software such as
OpenGL. We do not need these in a rigourous mathematical
definition. Likewise, in a HomogenousMesh we structure the realization as
follows: 1. Insert all vertices of the mesh into \texttt{vertices} 2. Construct
faces of at least 3 indices referencing the points in \texttt{vertices}.

This gives us certain properties that are nice for computation. Primarily
this allows us to observe the combinatorial properties of the mesh by analyizing
the faces. In addition, this compacts the data representation of vertices
since shared vertices can be represented with a common face index. Affine
transforms only need to operate on the vertices, and if it is closed and
faces share many vertices this may be up to 3 times faster.

However the most important issue with this type is that it is not
parameterized as a Polytope, and simply as a polyhedral mesh.

\section{Polytope Type}

We implemented a Polytope to address some of the issues with the
\texttt{HomogenousMesh} type. It is defined as follows:

\begin{lstlisting}
"""
A `Polytope` is an `N` dimensional object with elements `T` of the same type.
For example typealias `Polygon` and `Polyhedron` exist for dimensions 2 and
3 respectively.
"""
type Polytope{N,T} <: AbstractPolytope{N,T}
    elements::Vector{T}
end
\end{lstlisting}

The supertype \texttt{AbstractPolytope} type is not implied in the mathematical
sense, but rather to allow more granular definitions as needed for different
computational challenges. The \texttt{Polytope} type is parameterized
by \texttt{N}, the order of the polytope. The following
aliases exist for Polytopes with specified values for \texttt{N}:

\begin{lstlisting}
"""
A `Polygon` is a `Polytope` realizable with only two dimensions.
Generally this will be composed of `Points` or `LineSegment`s.
"""
typealias Polygon{T} Polytope{2,T}

"""
A `Polyhedron` is a `Polytope` realizable with only three dimensions.
Generally this will be composed of `Face`s or two-simplices (`Simplex{3}`).
"""
typealias Polyhedron{T} Polytope{3,T}
\end{lstlisting}

The final parameter, \texttt{T}, is the type of the elements. This may
simplify many representations, and allow more liberty in Polyhedron
representation. For example, constructions of polygons are straight forward
and may be a \texttt{Vector} of \texttt{Symbol} or \texttt{Point}.
However a Polyhedron may be constructed from \texttt{Simplex} or
\texttt{Polygon}. In this way it behaves as a wrapper of a \texttt{Vector}
with special type information associated. Of course, non-sensical constructions
may be made, but with sufficiently parameterized functions they will not
be operable.

\subsubsection{Functions}

Along with defining a \texttt{Polytope} we have added calculations for
area, volume, centroids, and various decomposition functions.

\url{https://github.com/JuliaGeometry/GeometryTypes.jl/pull/27}


\subsection{Signed Distance Fields}

A signed distance field (SDF) is a uniform sampling of an implicit function.
It was implemented earlier as a 
Below
we can see this in action over the definition of a circle.

\begin{lstlisting}
julia> f(x,y) = sqrt(x^2+y^2) - 1
f (generic function with 1 method)

julia> v = Array{Float64,2}(5,5) # construct a 2D 5x5 array of Float64

julia> for x = 0:4, y = 0:4
           v[x+1,y+1] = f(x,y)
       end

julia> v
5x5 Array{Float64,2}:
 -1.0  0.0       1.0      2.0      3.0    
  0.0  0.414214  1.23607  2.16228  3.12311
  1.0  1.23607   1.82843  2.60555  3.47214
  2.0  2.16228   2.60555  3.24264  4.0    
  3.0  3.12311   3.47214  4.0      4.65685
\end{lstlisting}

The results of \texttt{v} might be confusing since the matrix is oriented with
the origin in the top left corner. At coordinate $(0,0)$, or entry \texttt{v[1,1]},
we see that \texttt{f} is
equal to \texttt{-1}. Likewise we can see $(0,1)$ and $(1,0)$ are points on
the boundary since the value is \texttt{0} and everywhere else is positive.

Distance fields are interesting since they provide an intermediate representation
between functional space and discrete-geometric space. However they are
a very memory hungry data structure. We have created a data type called
\texttt{SignedDistanceField}.

\todo{example of the data type, and configuration space}


\section{Parametric Polyhedra}

The purpose of Parametric Polyhedra is to allow a polytope to
be represented with angles and edge lengths.


\subsection{ParametricTriangle}
In order for
us to start we must parameterize a triangle. Our first
definition is as follows:

\begin{lstlisting}
type ParametricTriangle{T}
    # edge lengths
    a::Nullable{T}
    b::Nullable{T}
    c::Nullable{T}
    # angles (radians)
    alpha::Nullable{T}
    beta::Nullable{T}
    gamma::Nullable{T}
end
\end{lstlisting}

It uses the \texttt{Nullable} type to give values the additional
property of being known or unknown. A \texttt{Nullable} often checked with the
\texttt{isnull} function, overloaded as follows:

\begin{lstlisting}
function Base.isnull(p::ParametricTriangle)
    isnull(p.a) || isnull(p.b) || isnull(p.c) ||
    isnull(p.alpha) || isnull(p.beta) || isnull(p.gamma)
end
\end{lstlisting}

In order to check the configuration space of the \texttt{ParametricTriangle} as
valid we needed to check all of the values are defined and
follow the sine and cosine relations:

\begin{lstlisting}

"""
Test if a ParametricTriangle has a valid configuration.
"""
function Base.isvalid(p::ParametricTriangle)
    # underdetermined case
    isnull(p) && return false
    # otherwise check constraints since all values exist
    a = get(p.a)
    b = get(p.b)
    c = get(p.c)
    alpha = get(p.alpha)
    beta = get(p.beta)
    gamma = get(p.gamma)
    return a*cos(beta) + b*cos(alpha) - c == 0 &&
           b*sin(alpha) - a*sin(beta) == 0 &&
           alpha + beta + gamma - pi == 0
end

# version with isapprox for floats
function Base.isvalid{T<:AbstractFloat}(p::ParametricTriangle{T};
                                        rtol=sqrt(eps(T)),
                                        atol=zero(T))
    # underdetermined case
    isnull(p) && return false
    # otherwise check constraints since all values exist
    a = p.a.value
    b = p.b.value
    c = p.c.value
    alpha = p.alpha.value
    beta = p.beta.value
    gamma = p.gamma.value
    return isapprox(a*cos(beta) + b*cos(alpha) - c,0,
                    rtol=rtol,atol=atol) &&
           isapprox(b*sin(alpha) - a*sin(beta),0,
                    rtol=rtol,atol=atol) &&
           isapprox(alpha + beta + gamma - pi,0,
                    rtol=rtol,atol=atol)
end
\end{lstlisting}

If some of the edge values in a triangle are unspecified, the following
function may complete the \texttt{ParametricTriangle}.

\begin{lstlisting}
"""
Given an underdetermined ParametricTriangle, compute the missing values
and return a new ParametricTriangle
"""
function Base.fill(p::ParametricTriangle)
    # all angles must be specified
    if isnull(p.alpha) || isnull(p.alpha) || isnull(p.gamma)
        error("Cannot fill in values for this triangle. All angles must be specified")
    end
    alpha = get(p.alpha)
    beta = get(p.beta)
    gamma = get(p.gamma)
    # no edges given, use circumcircle=1
    if isnull(p.a) && isnull(p.b) && isnull(p.c)
        e = edges(alpha,beta,gamma)
        return ParametricTriangle(e[1],e[2],e[3],p.alpha,p.beta,p.gamma)
    else
        # find the circumcircle
        D = !isnull(p.a) ? get(p.a)/sin(alpha) :
            !isnull(p.b) ? get(p.b)/sin(beta) :
            get(p.c)/sin(gamma) # one must be specified because of prior check
        # we only need to figure one side that is specified
        # so we can (re)compute the other two
        if !isnull(p.a)
            return ParametricTriangle(p.a, _edge(beta,D), _edge(gamma,D),
                                      p.alpha, p.beta, p.gamma)
        elseif !isnull(p.b)
            return ParametricTriangle(_edge(alpha,D), p.b, _edge(gamma,D),
                                      p.alpha, p.beta, p.gamma)
        elseif !isnull(p.c)
            return ParametricTriangle(_edge(alpha,D), _edge(beta,D), p.c,
                                      p.alpha, p.beta, p.gamma)
        end
    end
    # otherwise we broke it ... TODO remove once we start using this in loops
    error("filling triangle failed! What did you do???")
end
\end{lstlisting}


\subsection{ImplicitTriangle}

The purpose of the implict triangle is to use the law of sines to
validate a given triangle configuration. The law of sines is given
below:

\begin{equation}
\frac{a}{sin(A)} = \frac{b}{sin(B)} = \frac{c}{sin(C)} = d
\end{equation}

The common value, $d$, is the triangle's circumcircle diameter.
Thus if we are given 3 edge lengths ($a, b, c$) we may compute this
value directly
with the following:

\begin{equation}
d = \frac{2abc}{\sqrt{(a+b+c)(-a+b+c)(a-b+c)(a+b-c)}}
\end{equation}

Since we know edge lengths will be strictly positive, and sine is
positive in the range of 0 to $\pi$. The configuration space
may be mapped with the following function:

\begin{lstlisting}
function implicit_triangle(a,b,c,alpha,beta,gamma)
    r = a*b*c/sqrt((a+b+c)*(a-b+c)*(a+b-c)*(b+c-a))
    min(sin(alpha)/a - 2r,
        sin(beta)/b - 2r,
        sin(gamma)/c - 2r)
end
\end{lstlisting}

For example, this function may be mapped over the range of angle values
using a \texttt{SignedDistanceField}.

\begin{lstlisting}
using GeometryTypes

res = 0.1

s = SignedDistanceField(HyperRectangle(Vec(0,0.),Vec(pi*1,pi*1)), res) do v
    implicit_triangle(3,3,3,v[1],v[2],pi/3)
end
\end{lstlisting}

For our purposes a global minima search may be performed fairly quickly.
Other iterative techniques may also be used for solving the configuration
space such as gradient descent and the BFGS algorithim.






\chapter{Conclusion}

\section{Future Work}

\subsection{Automatic Differention of Solids}

\begin{lstlisting}
julia> using DualNumbers

julia> f(x) = 2x+1
f (generic function with 1 method)

julia> f(Dual(1,1))
3 + 2du
\end{lstlisting}



\subsection{Ray Tracing and Marching}

\todo{leave this section, would be good to discuss angle-based polyhedra
and or deficits for these ops}

When we look at the natural world we observe the
propogation of light energy. Our eyes recieve this light energy in the form
of photons. The study of ray tracing seeks to mimic such behavior for
computer visualizations and simulations. 

\begin{figure}[h!]
  \centering
    \includegraphics[width=0.75\textwidth]{img/ray_trace_diagram.png}
  \caption{An illustration of a Ray tracing.\protect\footnotemark}
  \label{fig:raytrace}
\end{figure}

\footnotetext{By Henrik (Own work) GFDL or CC BY-SA 4.0-3.0-2.5-2.0-1.0, via Wikimedia Commons}

Íñgo Quílez has done some of the most accessible work on real-time ray tracing.
His technique is called ray marching, and leverages the properties of functional
geometry.\cite{Quilez_2008}

\subsection{Engineering Solid Analysis}







\bibliography{references}

\end{document}

