
\chapter{Mathematical Definitions}

We are interested in constructing parameteric polytopes for
mathematical exploration. An informal geometric
picture of a polytope has already been developed.
In this chapter we will develop a mathematical perspective of our problem
such that we can focus clearly on the computational aspects in the following
sections. 


\section{Spacial Terminology}

\subsection{Vector Spaces}

Given a field (in the algebraic sense) of numbers, a vector of dimensionality
N is formed by an N-tuple of numbers in the field. A vector space must
be closed under element-wise addition with another vector and element-wise
multiplication by a scalar value. Symbolically, given $s \in F$ where $F$ is a
field, and $X,Y \in V$ where V is the vector space over $F$ then
$X+Y \in V$ and $s*x \in V$ implies our closure property.
In this paper we will primarily use
the real numbers in euclidean N dimensional space, denoted as $\mathbb{R}^N$.
The term "point" will generally be used to describe a vector that may
described using numerical values.

\subsubsection{Closed Vector Spaces}

\todo{start with proper definition}
A "closed" space means a subset of something that contains all points in it's
boundary, including the boundary.
\todo{boundary is ambiguous}
 An "open" set conversely will not include
the boundary, but everything inside.
We sometimes use brackets to aid the representation of sets on a number line
for example: (1,2) for an open set and
[1,2] for a closed set. The quantity of the objects contained in our sets
will vary depending on the numeric domain we choose.
For example if we are using integers we have no
elements in the open set and 2 in the closed set. If it is in rationals we
have 1/2 included, and in the reals we have infinite points! What we have
constructed on the number line can also be thought of as open
and closed intervals.


\subsection{Solids}

Before we define a polyhedra, we must introduce a few notions. These are
solids and orientation. Solids have been studied since
antiquity and for our purposes we will define them as constructions in
three dimensional space with finite volume.
For example, a plane which partitions space is not a solid
since either partition is unbounded, however the intersection of planes
could form a solid.

Orientation is a parallel concept which allows us to specify how geometric
objects contain space. As an example, let us go back to our partition of
space with a plane. If we are on our way to construct a solid, it is
neccessary to choose the one part to keep and the other to discard. This is
the purpose of orientation. In the case of the plane, this follows from the
definition, $ax+by+cz+d=0$. More lucidly, lets look at the signed distance
of a point from the plane, computed as:
\begin{equation}
D = \frac{ax+by+cz+d}{\sqrt{a^2+b^2+c^2}}
\end{equation}
For simplicity, let's look at a plane parallel to the X and Y axes passing
through z = 1. Thus $a$ and $b$ are set to 0, $c$ set to 1, and
$d=-1$
Simplifying our formulation we have:
\begin{equation}
D = 0x+0y+z-1 = z-1
\end{equation}
At $z=1$ we see we are on the plane, however at $z=0$ and $z=2$ we get -1 and 1
respectively. The sign of the distance is our indicator of orientation. We can
choose an arbitrary convention as to which partition we will count, but akin
to the "right hand rule" in physics, the normals of the partition must point
outwards. In the case of the plane the normal is the vector $(a,b,c)$, which
in our realization is the upwards vector $(0,0,1)$. Since the convention is
such that the normals point outward, the partition we would consider in a
solid is all points \emph{in the opposite direction of the normal}.
Also to note is the importance of sign in our distance function. We can exploit
this behavior to indicate containment when performing set operations on
spatial partitions. Negative values thus indicate a point inside the partition
of interest. We have chosen this convention for this paper due to
it's ubiquity in computation frameworks. In the field of computer graphics
using descrete polytopes, this is often referred to as "winding order".

\section{Combinatorial Representation of a Polytope}

\subsection{Simplices}

A simplex (plural simplices) in N dimensional space is the minimal set of
descrete points whose convex hull form a closed subset of dimensionality N.
It is often thought
of as the generalization of a tetrahedra into N dimensions. In one dimensional
space, this is a closed interval or line segment. In two dimensions it is the
triangle, and in three it is the tetrahedra. These are called a 1-simplex,
2-simplex, and 3-simplex respectively. We will represent them as 

\todo{set definition}


\begin{figure}[h!]
  \centering
    \includegraphics[width=0.4\textwidth]{img/800px-1-simplex_t0.png}
    \includegraphics[width=0.4\textwidth]{img/800px-2-simplex_t0.png}
    \includegraphics[width=0.4\textwidth]{img/800px-3-simplex_t0.png}
    \includegraphics[width=0.4\textwidth]{img/800px-4-simplex_t0.png}
  \caption{Examples of Simplices}
  \label{fig:simplices}
\end{figure}

We notice that the 1-simplex is formed by two 0-simplices, a 2-simplex is formed
by three 1-simplices, and the 3-simplex by four 2-simplices. The components of
these compositions are called "faces". The 0-face is often called a vertex
and the 1-face an edge. If we tabled the quantities of each M-face in an
N-simplex out, they form
Pascal's triangle and thus follow the binomial coefficient.

\begin{equation}
{N+1}\choose{M+1}
\end{equation}

Simplices will be our most basic geometry we use for formulating the descrete
form of a polytope. In fact, the two- and three-simplices are already
polytopes!

\subsubsection{Orientation}

Since we will eventually like to construct a functional representation
of a polytope from a combinatorial form, we must consider orientation.
In our case, we will consider how the construction is ordered. More
specifically, how the points are specified. Let us consider the most basic
case on a number line of an interval or 1-simplex. If we choose 1 and 2 to
form our closed interval we may express this as [1,2] or [2,1]. In this
case the result really would not change to any practical effect, and more so
[2,1] may be considered an incorrect representation in some dicisplines.
This example simply shows
how ordering may be communicated symbolically.

In the case of the 2-simplex it becomes more clear. Given three unique points
$a$, $b$, and $c$ specifying a triangle we have two possible orderings. Namely
a "clockwise" and "counterclockwise" orientation. We see in effect the orientation
of the edges is the same in the forumulations $[a,b,c]$, $[c,a,b]$, and $[b,c,a]$,
since the edges $[a,b]$, $[b,c]$, and $[c,a]$ are all formed identically.
However the formulations $[c,b,a]$, $[a,c,b]$, and $[b,a,c]$ form the edges
$[b,a]$, $[c,b]$, and $[a,c]$. This is opposite to the first formulation we
specified!

\begin{figure}[h!]
  \centering
    \includegraphics[width=0.75\textwidth]{img/Winding_order.png}
  \caption{The two possible orientations of a triangle given 3 points.}
  \label{fig:tri_orientation}
\end{figure}


Thus we may define the orientation of an simplex as a class of orderings taken
by even permutation (cite Hempel Thesis). More so we see that the orientation
of the edges follow from the orientation of the simplex. This property
will be important when we analyize the structure of Polyhedra.



\todo{John M. Lee Introduction to Topological Manifolds}
\cite{Lee_2011}

We now have the tools neccesary to construct a combinatorial representation
of a polytope. In the functional representation we were interested in
extracting geometric information about points and their distances from the
polytope. The combinatorial representation will help us observe the
relationships between the faces. To construct an N-polytope we will use 
N-1 simplices. using set 

\todo{Simplicial Complexes}

\subsection{Construction of the Combinatorial Representation}


\section{Implicit Functional Representations of Solids}

\subsection{Hyperplanes}

In multiple dimensions this is more interesting since we may construct
various, rather arbitrary, geometries to make a closed space.
One common example is a hyperplane. A hyperplane is simply a generalization
of the a plane into arbitrary dimensions, with the property it is
of dimensionality N-1. For example if we are in 2D space, our hyperplane
is a line since it partitions our space into two parts. Likewise in 3D
space this is a plane. If we define a hyperplane functionally using vector
notation we can extract some interesting properties.
For simplicity in this example let us assume we have a hyperplane which
cuts through the origin.
If we let 
$\vec{x}$ be an arbitrary point in $N$ dimensional space,
and $\vec{a}$ be the slopes of each axis, then one functional construction is
simply the dot product, $dot(\vec{a},\vec{x})$. If x is on the hyperplane
the function will be equal to zero. If it is not zero,
then we may determine which side of the partition the point lies on.
Thus as a set representation the hyperplane is:

\begin{equation}
\{f(x)=dot(\vec{a},\vec{x})|f(x)=0,x \in \mathbb{R}\}
\end{equation}

\todo{define hyperplane sooner}
We notice that the hyperplane cuts space, but does not create a closed
subspace. What we are after is a "solid", which is the composition of
such objects forming a closed space. In fact, this leads to a proper definition
of a closed set in topology. We will say that \emph{a set is closed if and only if it
contains all of its limits}\cite{007054235X}. In the case of the lone hyperplane
it is not bounded, so itself therefore not closed (by not having limits). Thus
we must compose hyperplanes to form closed spaces!


\todo{Implicit Surface and Distance Field}

If we were to functionally define the boundary of a solid a predicate of the
form $f(x) = 0$ would suffice, where the solid is defined by all $x$.
However we previously mentioned that it is
useful in computation to define a function that returns infomation about
the membership of a point in the solid. Further more we can define our functional
solid to return a value corresponding to the shortest distance to the boundary.\cite{Olah_2011}
These functions are sometimes called "implicit" forms in solid modelling, but
more precisely they generate a distance field.

A sketch of this behavior in one dimension
can be seen in Figure \ref{fig:implicit-sketch}.

\begin{figure}[h!]
  \centering
    \includegraphics[width=0.75\textwidth]{img/implicit_sketch.png}
  \caption{Number line illustrating the construction of an implicit signed function}
  \label{fig:implicit-sketch}
\end{figure}

\subsubsection{Affine Transforms}

Functional representations can naturally deal with affine
transforms\cite{Henderson_2002}. Given some transform associated with a
solid, one simply applies the inverse transform to check membership.
The key word here is "associated" since for our purposes we will define
geometries without consideration of their tranforms. This is one aspect
in which computation will aid us extensively. If we let $f$ be a functional
representation of a solid or spatial partition
which generates a signed distance field, $A$ be the transform of the object,
and $x$ be the point whose membership in the solid is in question. It then
follows that the distance field formed by $f$ transformed by $A$ is obtained
by $f(A^{-1}*x)$.

\begin{figure}[h!]
  \centering
    \includegraphics[width=0.5\textwidth]{img/Fractal_fern_explained.png}
  \caption{Number line illustrating the construction of an implicit signed function}
  \label{fig:implicit-sketch}
\end{figure}

\todo{Affines are groups}

\subsubsection{Orientation}


\subsection{Set operations on Distance Fields}


One can also compose distance fields with logical operations. 
Below are basic set operations defined for these functions using our
sign conventions:

\begin{equation*}
\cap : \mathtt{min}(f_1,f_2) \\
\end{equation*}
\begin{equation*}
\cup : \mathtt{max}(f_1,f_2) \\
\end{equation*}
\begin{equation*}
\neg : -\mathtt{f}_1
\end{equation*}

\todo{proof/cite}

It follows that the ``difference"
of $f_1$ and $f_2$ is the intersection of $f_1$ with the negation of $f_2$,
$\mathtt{max}(f_1,-f_2)$. Thus we may compose functional representations
of geometry using the language of set operations, namely, union, difference,
intersection, and negation.
The mathematical analyst might have trouble with these formulations because
such operations create discontinuities. To illucidate this problem we may
use the language of norms. A distance field as we have defined it behaves
like the inf-norm, or maximum norm. If we choose a point and take a projection
to the surface which forms the shortest path, our distance value
will be linear. This is not very useful for any kind of differential
analysis since we only have first order derivatives.
It may be useful for a parametric mathematical construction of a polytope
to have differentiable properties. Rvachev functions
provide one solution to this problem.


\subsubsection{Rvachev Functions}

In the 1960's Vladimir Rvachev produced a method for handling the "inverse
problem of analytic geometry". His theory consists of functions which provide a
link between logical and set operations in geometric modeling and analytic
geometry.\cite{shapiro1991theory} While attempting to solve boundary value problems,
Rvachev formulated an equation of a square as
\begin{equation*}
a^2 + b^2 − x^2 − y^2 + \sqrt[]{( a^2 − x^2 )^2 +( b^2 − y^2 )^2} =0
\end{equation*}

Implicitly, the sides of a square can be defined as $x= +/- a$ and $y= +/- b$.
The union of these two is a square. By reducing the formulation of the square
he generalized an expression for the union between two functions.
\begin{equation*}
\cup : f_1 + f_2 + \sqrt[]{f_1^2 +f_2^2} = 0
\end{equation*}

Likewise we can see that intersections and negations can be formed for logical
completion.
\begin{equation*}
\cap : f_1 + f_2 - \sqrt[]{f_1^2 +f_2^2} =0 \\
\end{equation*}
\begin{equation*}
\neg : -f_1
\end{equation*}

These formulations can be modified for $C^m$ continuity for any $m$.

\todo{clean}

The Rvachev construction is of hypothetical interest in the presentation,
but shows how geometric
constructions can converge with analytic constructions.

In 2000 Rvachev published an english-language proof of the completeness
\cite{Rvachev_Sheiko_Shapiro_Tsukanov_2000}

\cite{shapiro2007semi} In addition Pasko, et. al. have shown that Rvachev
functions can serve to replace a geometry kernel by creating logical
predicates. \cite{pasko1995function} Their research also establishes the
grounds for user interfaces and environment description. For this work a
practical implementation will most likely leverage their insights.
Rvachev and Shapiro have also shown that using the POLE-PLAST and SAGE
systems a user can generate complex semi-analytic geometry
as well.\cite{rvachev2000completeness} 

One of the most general expositions in the English language of R-Functions
applied to BVPs is
Vadim Shapiro's``Semi-Analytic Geometry with R-Functions". \cite{shapiro2007semi}
Unfortunately, no monographs about R-Functions exist in the English literature.
Most literature is in Russian, however many articles presenting applied
problems using the R-Function Method. \cite{voron2010}

Such a system for analytic geometry can be developed further. In the context
of an Eulerian flow field, a distance field over a function that
generates partial derivatives could be a fast numerical computation method.

\todo{clean}



\subsection{Construction of the Implicit Functional Representation}

We now have the mathematical concepts needed to define a polytope using the
language of hyperplanes. Using vector notation we may define a hyperplane 
as an implicit function generating a distance field by:

\begin{equation}
f(x) = \frac{a \cdot x }{\sqrt{a \cdot a}}
\end{equation}

where $a$ are the slopes and $x$ being the point in question. This function
will be zero when $x$ lies on the plane.

\cite{Polyhedra}

