\documentclass[a4paper]{article}

\usepackage[english]{babel}
\usepackage[utf8x]{inputenc}
\usepackage{amsmath}
\usepackage{graphicx}
\usepackage{listings}
\usepackage{hyperref}
\usepackage[square,sort,comma,numbers]{natbib}
\usepackage[colorinlistoftodos]{todonotes}
\bibliographystyle{ieeetr}

\lstdefinestyle{julia}{
  basicstyle=\scriptsize\ttfamily,
  breaklines=false,
  backgroundcolor=\color{gray!10},
}
\lstset{style=julia}

\title{A Research Proposal on Mathematically Rigorous and Computationally
Efficient Representations of Geometry}
\author{Stephen Kelly}

\begin{document}
\maketitle


\section{Introduction}

Geometry is one of the earliest academic studies in mathematics. Research is
constantly leading to new patterns and constructions. Many of these find
highly practical applications in engineering. In this paper we will seek to
study geometry by developing patterns that are sensible to computers and people.
To obtain sensiblile patterns we address two different audiences. One of
which is the computer which requires execution effiency. The other is people,
who require lucid representation of code.

This project will try to refine existing implementations of geometry for
computers and explore new procedures. As such, we will review the existing
systems, their applications, and areas for improvement. Later we will discuss
some of the areas for improvement in more detail, and establish a proposed
scope for this project.


\subsection{The Julia Programming Language}

Programming languages are the grammar and syntax a computer presents to a user.
This project is fundamentally exploratory in nature and seeks to generate
understanding of geometric relationships. I plan to use Julia as a programming
language for exploration. There are many reasons this language is ideally
suited for such mathematical exploration.

 Functions and Data
Julia is a descendent of the Lisp family of programming languages. Similar
to Lisp, it is an experiment in language design. Much of this advancement
revolves around the representation of data and the execution of functions.
The language optionally typed, which means function specialization on types
is inferred. See below:
\begin{lstlisting}
julia> increment(x) = x + 1
increment (generic function with 1 method)

julia> increment(1)
2

julia> increment(1.0)
2.0
\end{lstlisting}
the `increment` function was defined for any `x` value. When the `1`, an
integer type was passed as an argument, an integer was returned. Likewise
when a floating point, `1.0` was passed, the floating point `2.0` was returned.

Let's see what happens when we try a string:
\begin{lstlisting}
julia> increment("a")
ERROR: MethodError: `+` has no method matching +(::ASCIIString, ::Int64)
Closest candidates are:
  +(::Any, ::Any, ::Any, ::Any...)
  +(::Int64, ::Int64)
  +(::Complex{Bool}, ::Real)
  ...
 in increment at none:1
\end{lstlisting}

The problem is that the `+` function is not implemented between the
`ASCIIString` and `Int64` types. We need to either implement a `+` function
which might be ambiguous, or specialize the function for `ASCIIString`.
A specific implementation is preferrable in this case:
\begin{lstlisting}
julia> function increment(x::ASCIIString)
           ASCIIString([increment(c) for c in x])
       end
increment (generic function with 2 methods)
\end{lstlisting}
Since `ASCIIString` is a series of 8 bit characters, we can iterate over the
string and increment each character individually. The `[]` indicates we are
constructing an array of characters to pass to be passed to the `ASCIIString`
type constructor. Now we see our example works:
\begin{lstlisting}
julia> increment("abc")
"bcd"
\end{lstlisting}

What was demonstrated here is the concepts of specialization and multiple
dispatch, both are highly coupled topics.
Each function call in Julia is specialized for types if possible.
This means the author only has to write a few sufficently abstract
implementations of functions. If special cases occur multiple functions
with different arity or type signatures can be implmented. Explicitly
this is called multiple dispatch. In practice by the user this looks like
abstracted code. To the computer, this means choosing the most specific, and
thus performant method. Let's go back to the integer and floating point
example. Below is the LLVM assembly generated for each method:
\begin{lstlisting}
julia> @code_llvm increment(1)

define i64 @julia_increment_21458(i64) { // <return type> <function name>(<arg type>)
top:
  %1 = add i64 %0, 1
  ret i64 %1 // return <return type> <return id>
}

julia> @code_llvm increment(1.0)

define double @julia_increment_21466(double) {
top:
  %1 = fadd double %0, 1.000000e+00
  ret double %1
}
\end{lstlisting}

The only real similarity is the line count. Note I have annotated the LLVM code
so this is understandable. Each one of these functions are generated by the
Julia compiler at run time. The REPL (Read-Eval-Print-Loop) allows interactive
evaluation of Julia code. It is highly useful for exploration and testing of
ideas.

Many of the concepts used for performance also serve as methods for
expressability. In this case, multiple dispatch used by the compiler for
specialization of functions reveals it self as a way for the user to
specialize over many types.
Revealing the role in which this paradigm allows Julia to achieve high
performance is a matter to be developed in further sections.

 Types, Immutables, and Parameters

Types and immutables are containers of data. The primary difference between
the two is the notion of "mutability". Types are mutabile, immutables are 
immutable. What does this mean? Let's break something first:
\begin{lstlisting}
julia> type FooIsMutable
           a
       end

julia> f = FooIsMutable(1)
FooIsMutable(1)

julia> f.a
1

julia> f.a = 2
2

julia> f.a
2

julia> immutable FooIsImmutable
           a
       end

julia> f = FooIsImmutable(1)
FooIsImmutable(1)

julia> f.a
1

julia> f.a = 2
ERROR: type FooIsImmutable is immutable
\end{lstlisting}

What just happened demonstrates the contract defined by mutability. Mutable
objects, which is an instance of a type (i.e. `f`), can have their fields
(i.e. `a`) changed. Immutables cannot. The immutable contract helps develop
a notion of functional purity. Practically this can be of great benefit to
the compiler. For example:
\begin{lstlisting}
julia> a = (1,2,3)
(1,2,3)

julia> b = typeof(a)
Tuple{Int64,Int64,Int64}

julia> isbits(b)
true

julia> a = ([1],[2],[3])
([1],[2],[3])

julia> b = typeof(a)
Tuple{Array{Int64,1},Array{Int64,1},Array{Int64,1}}

julia> isbits(b)
false
\end{lstlisting}

`isbits` ask the question "will this type be tightly packed in memory"? A
`Tuple` is a fixed-length set of linear, ordered, data. It has syntax for
construction with `()`. In computations we want our data be close together
for fast access. In modern times we call such data "cache friendly", or
"cache localized". Immutability helps us achieve this. Let's look that the
types inside the 3-tuples and see their `isbits` status:
\begin{lstlisting}
julia> isbits(Array{Int64,1})
false

julia> isbits(Int64)
true
\end{lstlisting}
Why is this the case? We see that `Int64` is bits, because it is literally
64 bits. `Array{Int,64}` is a data type 

 Solid Modeling Paradigms

The expression of solid bodies is fundamental in the development of any
natural problem statement. For example, in diffusion we model the transfer of
energy throughout a domain. An engineer might define such a domain with a
model, say of an injection molding nozzle. Such a domain is difficult to
describe in terms of a functional boundary. The development of modern
computational tools for solid modeling have vastly different paradigms.
Many 

 Functional Representation

Functional representation in computation centers around a signed, real-value
function where the boundary is defined as `f(...) = 0`.
In `R3` this looks like `f(x,y,z) = 0`. For modelling purposes we must add the
additional constraint that the function evaluates to a negative inside the
boundary. Further more the magnitude of the return value must correspond to
the minimum distance between the point and the boundary.


 Mesh


 BRep


 CSG



 Exploration

 Geometry Types

GeometryTypes.jl is a package for Julia that provides geometric strutures and
relations. It was started early 2015 as the integration of Meshes.jl,
ImmutableArrays.jl, HyperRectangles.jl, and FixedSizeArrays.jl. This package
was able to resolve the relations between geometric structures. With the
release of Julia version 0.4 is became possible to build the appropriate
abstractions. For example ImmutableArrays represented a 3 dimensional
vector with the concrete type `Vector3{Int64}`. FixedSizeArrays introduced
the dimensionality as a parameter as `Vector{3,Int64}`. This means the notion
of a fixed length vector can be abstracted over arbitrary dimensionality.

 Simplices

Recently a `Simplex` type was added to `GeometryTypes`. A `Simplex` is defined
as the minimum convex set containing the specified points. The initial
prototype


 Distance Fields



 Dual Numbers


 Rvachev Functions


In the 1960's Vladimir Rvachev produced a method for handling the "inverse
problem of analytic geometry". His theory consists of functions which provide a
link between logical and set operations in geometric modeling and analytic
geometry.\cite{shapiro1991theory} I believe the following anecdote helps
elucidate the theory. While attempting to solve boundary value problems,
Rvachev formulated an equation of a square as
\begin{equation*}
a^2 + b^2 − x^2 − y^2 + \sqrt[]{( a^2 − x^2 )^2 +( b^2 − y^2 )^2} =0
\end{equation*}

Implicitly, the sides of a square can be defined as $x= +/- a$ and $y= +/- b$.
The union of these two is a square. By reducing the formulation of the square
we can generalize an expression for the union between two functions.
\begin{equation*}
\cup : f_1 + f_2 + \sqrt[]{f_1^2 +f_2^2} =0
\end{equation*}

Likewise we can see that intersections and negations can be formed for logical
completion.
\begin{equation*}
\cap : f_1 + f_2 - \sqrt[]{f_1^2 +f_2^2} =0 \\
\end{equation*}
\begin{equation*}
\neg : -f_1
\end{equation*}

These formulations can be modified for $C^m$ continuity for any $m$.
\cite{shapiro2007semi} In addition Pasko, et. al. have shown that Rvachev
functions can serve to replace a geometry kernel by creating logical
predicates. \cite{pasko1995function} Their research also establishes the
grounds for user interfaces and environment description. For this work a
practical implementation will most likely leverage their insights.
Rvachev and Shapiro have also shown that using the POLE-PLAST and SAGE
systems a user can generate complex semi-analytic geometry
as well.\cite{rvachev2000completeness} 
\subsection{Numerical Analysis}
While a functional representation for geometry is mathematically enticing on
its own, the power it gives for numerical analysis might be its greatest
virtue. Numerical analysis justified the initial investigation by Rvachev
early on. A boundary value problem on a R-Function-predicate domain allows
for analysis without construction of a discrete mesh.\cite{rvachev2000completeness}

One of the most general expositions in the English language of R-Functions
applied to BVPs is
Vadim Shapiro's``Semi-Analytic Geometry with R-Functions". \cite{shapiro2007semi}
Unfortunately, no monographs about R-Functions exist in the English literature.
Most literature is in Russian, however many articles presenting applied
problems using the R-Function Method. \cite{voron2010} This is the topic
of this project I stand to gain the most insight.


 Mesh Slicing




Today the dominant paradigm for solid modeling has remained unchanged since the 1970's, relying primarily on Boundary Representation (B-Rep). It relies primarily on the manipulation and representation of edges, vertices, and faces to build a model. The primary mechanism for the representation is a "feature tree". While B-Rep is intuitive for users of a graphical environment, it is unwieldy as a textual and functional representation.  This methods is natural for engineers and designers, but sacrifices parametric design. In addition, B-Rep requires the use of a geometry kernel to handle the interpretation of constraints and geometric construction. \cite{stroud2006boundary}

Geometry kernels often decouple functional representations from a user's design hierarchy which complicates numerical analysis.\cite{lee2005cad} This middle step of Computer Aided Engineering (CAE) is known as pre-processing. For example in the Finite Element Analysis (FEA) process the requires establishing proper aspect ratio, area, and connectivity of nodes. Research has shown that functional representations can simplify or eliminate these steps and algorithms.

In addition, designers targeting parametric design have turned to the methods of Constructive Solid Geometry (CSG), which works using manipulation of geometric primitives (half-spaces) as a level of abstraction. This enables parametric solids to be represented using operations and relations on primitive solids. CSG has been growing in popularity due to programs such as OpenSCAD\footnote{\url{http://www.openscad.org}}, CoffeeSCAD\footnote{\url{http://coffeescad.net/}}, POVRay\footnote{\url{http://www.povray.org/}} and Thingiverse Customizer\footnote{\url{http://www.thingiverse.com}}. These programs are particularly popular for collaboration in conjunction with version control systems such as Git.

\subsection{Closed Loop Optimization}
Optimization could be studied using mathematical solid modeling. I have not completely researched existing literature on this topic, so I include this idea as a ``stretch goal" for the project. Given a parametrized solid and numerical analysis, optimization might prove useful. Possible ideas include the minimization of mechanical stress on a solid or reduction of material volume.


\section{Technologies}
\subsection{Julia}
Julia is a programming language first released in early 2012 by a group of developers from MIT. The language targets technical computing by providing a dynamic type system with near-native code performance. This is accomplished by using three concepts: a Just-In-Time (JIT) compiler to target the LLVM framework, a multiple dispatch system, and code specialization.\cite{bezanson2012julia} The syntactical style is similar to MATLAB and Python. The language implementation and many libraries are available under the permissive MIT license.\footnote{\url{http://opensource.org/licenses/MIT}}

Benchmarks have shown the language can consistently perform within a factor of two of native C and FORTRAN code.\footnote{\url{http://julialang.org/benchmarks}} This is enticing for a solid modeling application and for numerical analysis, as the code abstraction can grow organically without performance penalty. In fact, the authors of Julia call this balance a solution to the ``two language problem". The problem is encountered when abstraction in a high-level language will disproportionately affect performance unless implemented in a low-level language. 

Since February 2014 I have been actively contributing to the Julia community through contributions to the base compiler and library, package maintenance and development, and general community support through forums and mailing lists. From September 2014 to May 2015 I developed a multiple material path planner in Julia for 3D printers at my co-op at Voxel8. There I learned to architect and optimize Julia code. When I first discovered Julia I realized it had great potential for CAD, and the
I would like to use Julia as a media of exploration for the mathematical concepts and techniques during this project. While the language is relatively new and in development, sufficient documentation and an active community can help support software development. Many advanced and mature packages for fields such as operations research exist.\cite{lubin2013computing}


\subsection{Open Source}
Along with Julia, I believe it is important to publish software frequently and openly. In the past the open source community has been very receptive to new ideas. I expect that over the course of the project we can attract external contributions to the software developed. This will help strengthen and quicken developments.

As such, I believe it will be important to improve the accessibility to the mathematical theory. Along with a report, I would like to improve and write Wikipedia pages to enable a greater audience. The existing Wikipedia page for Rvachev-Functions\footnote{\url{http://en.wikipedia.org/wiki/Rvachev_function}} contains simple concepts compared to the richness of the academic literature existing on the topic.


\section{Conclusion}
The motivation of this project is social. 3D printing has promise to be disruptive to economics and technological application. However it is only the physical media in which collaboration between people can change. The underlying mathematics of the \emph{digital} media for collective action will be the focus of this project. 

The intellectual challenge for this project will be mathematical.  In particular, the question is \emph{how can techniques for functional solid modeling and numerical analysis be applied to improve the state of 3D modeling?} My previous insight into the importance of parametric design and engineering analysis will drive the development of a solid mathematical understanding.

The application of this project will be through distillation of mathematical theory into software and documentation. In particular, libraries and scripts will provide means for others to experiment and improve our work. Documentation and explanations of theory will enrich the open source 3D modeling community. 

I first encountered Rvachev functions in October 2013 while researching mathematical techniques to implement a CAD program. The existing literature is vast, however few practical implementations exist for the maker community to use. I expect this project to take 4/3 units of work during the 2014-2015 academic year. My hope is this will permit me to review the literature and develop applications of utility that might inspire more research. I appreciate your time in reviewing this proposal.

\bibliography{references}

\end{document}

