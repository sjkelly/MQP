\documentclass[a4paper]{article}

\usepackage[english]{babel}
\usepackage[utf8x]{inputenc}
\usepackage{amsmath}
\usepackage{graphicx}
\usepackage{xcolor}
\usepackage{listings}
\usepackage{hyperref}
\usepackage{amssymb}
\usepackage[square,sort,comma,numbers]{natbib}
\usepackage[colorinlistoftodos]{todonotes}
\bibliographystyle{ieeetr}

\lstdefinestyle{julia}{
  basicstyle=\scriptsize\ttfamily,
  breaklines=false,
  backgroundcolor=\color{gray!10},
}
\lstset{style=julia,
        emph={julia},
        emphstyle={\color{green}\bfseries}}


\title{A Research Proposal on Mathematically Rigorous and Computationally
Efficient Representations of Geometry}
\author{Steve Kelly}

\begin{document}
\maketitle

\begin{abstract}
Geometry is fundamental to the development of natural problem statements.
Opportunities exist in computational geometry to
add rigourous type descriptions and algorithmic relations that do not
sacrifice performance. This proposal will develop a foundational context
for such a research project.
\end{abstract}

\section{Introduction}

Geometry is one of the earliest academic studies in mathematics. Research is
constantly leading to new patterns and constructions. Many of these find
highly practical applications in engineering. In this paper we will seek to
study geometry by developing patterns that are sensible to computers and people.
To obtain sensible patterns we address two different audiences.
One is the computer which requires execution efficiency. The other is people,
who require lucid representation of code. Such challenges are well known in
the Human-Computer Interaction (HCI) field. \cite{Tognazzini_1993}

This proposal will try to refine existing implementations of geometry for
computers and explore new procedures. As such, we will review the existing
systems, their applications, and areas for improvement. Later we will discuss
some of the areas for improvement in more detail, and establish a proposed
scope for the project going forward.

\section{The Julia Programming Language}

Programming languages are the grammar and syntax a computer presents to a user.
This project is fundamentally exploratory in nature and seeks to generate
understanding of geometric relationships. I will use Julia as a programming
language for exploration. In the following sections I hope to develop some
rationale for this choice, and give a brief introduction to the language
concepts that will help make computationally effective geometry representations
possible.

In addition we will use it to illustrate some concepts and relations.
The REPL (Read-Eval-Print-Loop) allows interactive
evaluation of Julia code. It is highly useful for exploration and testing of
ideas in the language.
Blocks starting with "\texttt{julia>}" represent input and the preceding
line represents output of the evaluated line.

\subsection{History}
Julia is a programming language first released in early 2012 by a group of
developers from MIT. The language targets technical computing by providing a
dynamic type system with near-native code performance. This is accomplished by
using three concepts: a Just-In-Time (JIT) compiler to target the LLVM framework,
a multiple dispatch system, and code specialization.\cite{bezanson2012julia}
\cite{Bezanson_Edelman_Karpinski_Shah_2014}
The syntactical style is similar to MATLAB and Python.
The language implementation and many libraries are available under the
permissive MIT license.\footnote{\url{http://opensource.org/licenses/MIT}}

Benchmarks have shown the language can consistently perform within a factor of
two of native C and FORTRAN code.\footnote{\url{http://julialang.org/benchmarks}}
This is enticing for a solid modeling application and for numerical analysis,
as the code abstraction can grow organically without performance penalty.
In fact, the authors of Julia call this balance a solution to the 
``two language problem". The problem is encountered when abstraction in a
high-level language will disproportionately affect performance unless
implemented in a low-level language. In the next sections we will compare
the expressibility and performance to other languages.

\subsection{Comparisons}

Many languages are as fast as Julia but sacrifice expressibility.
In Figure \ref{fig:juliabench} we can see some comparisons to other programming
languages. This was developed by the Julia core team, and illustrates that
Julia is highly competitive in performance. Again, these results stem from
the compiler and language design. In Figure \ref{fig:juliaexpr} we can see
these results normalized against code length. The Julia code is quite short,
yet consistently achieves good performance.
Much of this comes down to the innovated type and function system. We will
discuss these more in depth later.

\begin{figure}[h!]
  \centering
    \includegraphics[width=1.0\textwidth]{img/juliabench.pdf}
  \caption{A comparison of programming languages and performance.}
  \label{fig:juliabench}
\end{figure}

\begin{figure}[h!]
  \centering
    \includegraphics[width=1.0\textwidth]{img/expressability.pdf}
  \caption{The results in Figure \ref{fig:juliabench} normalized for code length. (Courtesy of Simon Danish)}
  \label{fig:juliaexpr}
\end{figure}


In 1972 Alan Kay introduced the terms
"class" and "object", to describe a coupling of data and functionality.\footnote{\url{http://gagne.homedns.org/~tgagne/contrib/EarlyHistoryST.html}}
An object is simply and implementation of a class. Computer Scientists
call this "Object Oriented Programming" (OOP).
Languages such as C++, Java, and Python all subscribe to this paradigm.
In Python this looks like the following:
\begin{lstlisting}
class Foo:
    foo1
    foo2
    def add_to_foo1(self, x):
        self.foo1 += x
\end{lstlisting}

This system positively enables specialization of functionality, but due
to the coupling of data with functions it becomes a challenge to extend
functionality. Languages for scientific computing generally avoid the
"traditional" notions
of OOP. In Table \ref{tab:types} we can see a
comparison of type systems used in scientific computing languages. In the next
few sections the implications of multiple dispatch and the relation to OOP
will be developed further.


\begin{figure}[h!]
  \centering
    \caption{A comparison of functions, typing, and dispatch.}
    \begin{tabular}{ l | l l l}
    Language & Type system & Generic functions & Parametric types \\
    \hline
    Julia & dynamic & default & yes \\
    Common Lisp & dynamic & opt-in & yes (but no dispatch) \\
    Dylan & dynamic & default & partial (no dispatch) \\
    Fortress & static & default & yes \\
    \end{tabular}
  \label{tab:types}
\end{figure}


\subsection{Functions}
Julia is an experiment in language design. Much of the advancement
revolves around the representation of data and the execution of functions.
The language is optionally typed, which means function specialization on types
is inferred. We will use it See below:
\begin{lstlisting}
julia> increment(x) = x + 1
increment (generic function with 1 method)

julia> increment(1)
2

julia> increment(1.0)
2.0
\end{lstlisting}
the \texttt{increment} function was defined for any \texttt{x} value. When the
\texttt{1}, an
integer type was passed as an argument, an integer was returned. Likewise
when a floating point, \texttt{1.0} was passed, the floating point
\texttt{2.0} was returned.

Let's see what happens when we try a string:
\begin{lstlisting}
julia> increment("a")
ERROR: MethodError: `+` has no method matching +(::ASCIIString, ::Int64)
Closest candidates are:
  +(::Any, ::Any, ::Any, ::Any...)
  +(::Int64, ::Int64)
  +(::Complex{Bool}, ::Real)
  ...
 in increment at none:1
\end{lstlisting}

The problem is that the \texttt{+} function is not implemented between the
\texttt{ASCIIString} and \texttt{Int64} types.
We need to either implement a \texttt{+} function
which might be ambiguous, or specialize the function for \texttt{ASCIIString}.
A specific implementation is preferrable in this case:
\begin{lstlisting}
julia> function increment(x::ASCIIString)
           ASCIIString([increment(c) for c in x])
       end
increment (generic function with 2 methods)
\end{lstlisting}
The line \texttt{x::ASCIIString} is called a ``type annotation" and
states that \texttt{x} must be a subtype
of \texttt{ASCIIString}. This allows one to control dispatch of functions,
since Julia will default to the \emph{most specific implementation}.
Since\texttt{ASCIIString} is a series of 8 bit characters, we can iterate over the
string and increment each character individually. The \texttt{[]} indicates we are
constructing an array of characters to pass to be passed to the \texttt{ASCIIString}
type constructor. Now we see our example works:
\begin{lstlisting}
julia> increment("abc")
"bcd"
\end{lstlisting}

What was demonstrated here is the concepts of specialization and multiple
dispatch, both are highly coupled topics.
Each function call in Julia is specialized for types if possible.
This means the author only has to write a few sufficently abstract
implementations of functions. If special cases occur multiple functions
with different arity or type signatures can be implmented. Explicitly
this is called multiple dispatch. In practice by the user this looks like
abstracted or generic code.
To the computer, this means choosing the most specific, and
thus performant method. Let's go back to the integer and floating point
example. Below is the LLVM assembly generated for each method:
\begin{lstlisting}
julia> @code_llvm increment(1)

define i64 @julia_increment_21458(i64) { // <return type> <function name>(<arg type>)
top:
  %1 = add i64 %0, 1
  ret i64 %1 // return <return type> <return id>
}

julia> @code_llvm increment(1.0)

define double @julia_increment_21466(double) {
top:
  %1 = fadd double %0, 1.000000e+00
  ret double %1
}
\end{lstlisting}

Note I have annotated the LLVM code so this is understandable. 
The only real similarity is the line count. Each one of these functions are generated by the
Julia compiler at run time.

Many of the concepts used for performance also serve as methods for
expressability. In this case, multiple dispatch used by the compiler for
specialization of functions reveals it self as a way for the user to
specialize over many types.
Revealing the role in which this paradigm allows Julia to achieve high
performance is a matter to be developed in further sections.

\subsection{Types}

\subsubsection{Mutability and data packing}
Types and immutables are containers of data. The primary difference between
the two is the notion of "mutability". Types are mutabile, immutables are 
immutable. What does this mean? Let's break something first:
\begin{lstlisting}
julia> type FooIsMutable
           a
       end

julia> f = FooIsMutable(1)
FooIsMutable(1)

julia> f.a
1

julia> f.a = 2
2

julia> f.a
2

julia> immutable FooIsImmutable
           a
       end

julia> f = FooIsImmutable(1)
FooIsImmutable(1)

julia> f.a
1

julia> f.a = 2
ERROR: type FooIsImmutable is immutable
\end{lstlisting}

What just happened demonstrates the contract defined by mutability. Mutable
objects, which is an instance of a type (i.e. \texttt{f}), can have their fields
(i.e. \texttt{a}) changed. Immutables cannot. The immutable contract helps develop
a notion of functional purity. To the user this means immutables are defined
by their values. Practically this can be of great benefit to
the compiler. For example:
\begin{lstlisting}
julia> a = (1,2,3)
(1,2,3)

julia> b = typeof(a)
Tuple{Int64,Int64,Int64}

julia> isbits(b)
true

julia> a = ([1],[2],[3])
([1],[2],[3])

julia> b = typeof(a)
Tuple{Array{Int64,1},Array{Int64,1},Array{Int64,1}}

julia> isbits(b)
false
\end{lstlisting}

\texttt{isbits} ask the question ``will this type be tightly packed in memory"? A
\texttt{Tuple} is a fixed-length set of linear, ordered, data. It has syntax for
construction with \texttt{()}. In computations we want our data be close together
for fast access. In modern times we call such data "cache friendly", or
"cache localized". Immutability helps us achieve this. Let's look that the
types inside the 3-tuples and see their \texttt{isbits} status:
\begin{lstlisting}
julia> isbits(Array{Int64,1})
false

julia> isbits(Int64)
true
\end{lstlisting}
Why is this the case? We see that \texttt{Int64} is bits, because it is literally
64 bits. In Julia a \texttt{bitstype} behaves similar to an immutable, and is identified
by value. \texttt{Array\{Int,64\}} is a mutable data type that can vary in size.
This means
the \texttt{Tuple} needs to store the arrays as references, in this case a
pointer. When iterating over a data set, such a ``pointer dereferences" (this is
jargon for accessing the data in memory pointed to by a pointer), can be costly.
Modern CPUs accell when data is linearly packed and pointer-free. The
data can be brought into the CPU's memory cache once and computed without
shuffling between cache and RAM.

\subsubsection{Parameters}

\todo{need to demonstrate why this is HUGELY important for performance and expression}

\subsubsection{Macros and Generated Functions}
Julia is a descendant of the Lisp family of programming languages. Lisp
is a portmanteau for "List Processing". The language was designed to address
the new notion of "types", specifically in application to Artificial
Intelligence (AI) problems.\cite{McCarthy_1966} The notion of an "S-Expression"
was introduced in McCarthy's seminal work. These statements use parenthesis
to denote functions and arguments. Below is an an example of S-Expressions
for addition and multiplication.

\begin{lstlisting}
> (+ 1 1)
2

> (* 3 4)
12
\end{lstlisting}

This syntax is noted for it's mathematical purity.
However it can be a syntactic difficulty for many.
Most of the current popular programming languages
use variants of ALGOL syntax, which is noted for being more readable.
\cite{Hoare}
Julia also uses ALGOL syntax, but is lowered to S-Expressions. This enables
many of the mathematically pure relations we seek to achieve.
In addition S-Expressions are highly conducive to source transforms.
This develops a notion of "Homoiconicity", where the representation of
program structure is similar to the syntax. In Julia we use this property
to make "macros" which enable source code to be transformed based on
structure before compilation.

Generated functions perform a similar function as macros, but at the function
level. They enable source code to be procedurally generated based on types.
Surveys of Computer Science literature show that such a concept is new.
\todo{More on generated functions. Not sure if they should even be mentioned
since they are usually unnecessary}

\todo{Become more articulate like Graydon Hoare: \url{http://graydon2.dreamwidth.org/189377.html}}

\subsection{Example}


\subsection{On the changing state of Computation}

\cite{Shamos_1999}

\section{Solid Modeling Paradigms}

The expression of solid bodies is fundamental in the development of any
natural problem statement. For example, in diffusion we model the transfer of
energy throughout a domain. An engineer might define such a domain with a
model, say of an injection molding nozzle. Such a domain is difficult to
describe in terms of a functional boundary, so the engineer might prefer
a boundary representation.

In addition to being fundamental to natural studies, solid modeling is growing
in popularity due to low-cost digital manufacturing tools reaching the market.
There have been 3D printers popping up in nearly every educational
institutions over the past 3 years. In addition CNC routing, and laser cutting
enable people to go quickly from design to fabrication.

The development of modern
computational tools for solid modeling have vastly different paradigms. In
the next few sections we will layout the mathematical and computational
principles of these paradigms.

\subsection{Implicit Functional Representation}

Implicit functional representation (FRep) in computation centers around a signed, real-value
function where the boundary is defined as $f(...) = 0$.
In $\mathbb{R}3$ this looks like $f(x,y,z) = 0$. For modeling purposes we must add the
additional constraint that the function evaluates to a negative inside the
boundary. Further more the magnitude of the return value must correspond to
the minimum distance between the point and the boundary.\cite{Olah_2011}
A sketch of this behavior in one dimension
can be seen in Figure \ref{fig:implicit-sketch}.

\begin{figure}[h!]
  \centering
    \includegraphics[width=0.75\textwidth]{img/implicit_sketch.png}
  \caption{Number line illustrating the construction of an implicit signed function}
  \label{fig:implicit-sketch}
\end{figure}

Researchers at MIT have taken these principles to constrain
geometric features to ensure a part can be manufactured.\cite{Shugrina_Shamir_Matusik_2015}
They call their system ``FabForms" and shows how functional representations
can easily accept constraints.

More importantly, functional representations can naturally deal with affine
transforms. \cite{Henderson_2002} Given some transform associated with a
FRep, one simply applies the inverse transform to check membership.

One can also compose functional representations with set operations. The union
of two functions, $f_1$ and $f_2$ can be expressed by $min(f_1,f_2)$. An
intersection is expressed as $max(f_1,f_2)$, and it follows that the difference
of $f_1$ and $f_2$ is the intersection of the negation of $f_2$, $max(f_1,-f_2)$.
The mathematical analyst might have trouble with this because such operations
create discontinuities. This is one area of exploration that will be discussed
later.

\subsection{Signed Distance Fields}

A signed distance field (SDF) is a uniform sampling of an implicit function,
or any oriented geometry. \todo{didn't introduce orientation, winding order, etc...}
Below
we can see this in action over the definition of a circle.

\begin{lstlisting}
julia> f(x,y) = sqrt(x^2+y^2) - 1
f (generic function with 1 method)

julia> v = Array{Float64,2}(5,5) # construct a 2D 5x5 array of Float64

julia> for x = 0:4, y = 0:4
           v[x+1,y+1] = f(x,y)
       end

julia> v
5x5 Array{Float64,2}:
 -1.0  0.0       1.0      2.0      3.0    
  0.0  0.414214  1.23607  2.16228  3.12311
  1.0  1.23607   1.82843  2.60555  3.47214
  2.0  2.16228   2.60555  3.24264  4.0    
  3.0  3.12311   3.47214  4.0      4.65685
\end{lstlisting}

The results of \texttt{v} might be confusing since the matrix is oriented with
the origin in the top left corner. At coordinate $(0,0)$, or entry \texttt{v[1,1]},
we see that \texttt{f} is
equal to \texttt{-1}. Likewise we can see $(0,1)$ and $(1,0)$ are points on
the boundary since the value is \texttt{0} and everywhere else is positive.

Distance fields are interesting since they provide an intermediate representation
between functional space and discrete-geometric space. However they are
a very memory hungry data structure. Pixar has published OpenVDB which helps
work around these concerns, but such compression can be lossy.\cite{OpenVDB}
With the advent of shader pipelines for GPUs, distance fields have become
more popular. Valve has used SDFs with great success for generating smooth
text.text renders. \cite{Green_2007}
Many algorithms for generating polyhedra from an SDF
exist. The most common are Marching Tetrahedra, Marching Cubes,
and Dual Contours.\cite{Muller_Wehle_1997}\cite{Newman_Yi_2006}\cite{Cook_Hourvitz}

\todo{Talk about vert and frag shaders.}

Andreas Bærentzen and Henrik Aanæs published methods on the inverse
problem of converting a mesh to a signed distance fields.\cite{Baerentzen_Aanaes}
DiFi was introduced in 2004, which demonstrates an algorithm for creating
SDFs on multiple types of geometry \cite{Sud_Otaduy_Manocha_2004}.
\todo{Need to re-read this paper}

Many necessary algorithms in path planning for digital manufacturing tools
fall out of distance fields. For example, offsetting simply becomes
an addition or subtraction over the SDF. Computing the medial axis becomes
a scan for inflection points. Many path planners need to simplify polygon
representations as to not generate move less than the resolution of the machine.
Assuming the machine uses a Cartesian system, a SDF can correspond perfectly
to the lowest available resolution of the machine.
Likewise as Stereolithographic 3D printers
begin to use digital mirror devices (commonly known as DLP or DMD)
, discrete representations of geometry will become more important in
digital manufacturing.


\subsection{Mesh}

\cite{Heckbert_Garland_1997}

\cite{Bischoff_Botsch_Steinberg_Bischoff_Kobbelt_Aachen_2002}

\subsection{Boundary Rep}
Boundary Representation (B-Rep) has been the dominant modeling paradigm
for engineering since the 1970's.
It relies primarily on the manipulation and representation of
edges, vertices, and faces to build a model.
The primary mechanism for the representation is a "feature tree".
While B-Rep is intuitive for users of a graphical environment,
it is unwieldy as a textual and functional representation.
This methods is natural for engineers and designers, but sacrifices
parametric design. In addition, B-Rep requires the use of a geometry kernel
to handle the interpretation of constraints and geometric construction. \cite{stroud2006boundary}

Geometry kernels often decouple functional representations from a user's design
hierarchy which complicates numerical analysis.\cite{lee2005cad}
This middle step of Computer Aided Engineering (CAE) is known as pre-processing.
For example in the Finite Element Analysis (FEA) process the requires
establishing proper aspect ratio, area, and connectivity of nodes.

\todo{need more substance on how this is different from Mesh and the others}

\subsection{Constructive Solid Geometry}

Constructive Solid Geometry (CSG), works using manipulation of
geometric primitives (half-spaces) as a level of abstraction in design.
CSG has been growing in popularity due
to programs such as OpenSCAD\footnote{\url{http://www.openscad.org}},
CoffeeSCAD\footnote{\url{http://coffeescad.net/}},
POVRay\footnote{\url{http://www.povray.org/}}
and Thingiverse Customizer\footnote{\url{http://www.thingiverse.com}}.
Each exclusively uses text representations to describe geometry and all
are open source.
These programs are particularly popular for collaboration
in conjunction with version control systems such as Git.
Figure \ref{fig:csgtree} illustrates the process of describing complex
geometries with CSG.

\begin{figure}[h!]
  \centering
    \includegraphics[width=0.75\textwidth]{img/csg_tree.png}
  \caption{An illustration of a CSG tree.}
  \label{fig:csgtree}
\end{figure}

Most of these systems use mesh or functional representations as the
underlying method of operation.

\subsection{Graphs}

Constraints. Underlying mechanism of Brep.

Laplacian Contractions!!!
\cite{Cao_Tagliasacchi_Olson_Zhang_Su_2010}

\subsection{Linear Algebraic Representation}
\cite{DiCarlo_Paoluzzi_Shapiro_2014}

\subsection{Visualization}
\cite{Tupper_2001}
\cite{Hijazi_Knoll_Schott_Kensler_Hansen_Hagen_2008}

\section{Exploration}

\subsection{Rigorous Definitions of Geometry}

GeometryTypes.jl is a package for Julia that provides geometric structures and
relations. It was started early 2015 as the integration of Meshes.jl,
ImmutableArrays.jl, HyperRectangles.jl, and FixedSizeArrays.jl. This package
is able to resolve the relations between geometric structures for
computational algorithms and fast visualization on the GPU. With the
release of Julia version 0.4 is became possible to build the appropriate
abstractions. For example ImmutableArrays represented a 3 dimensional
vector with the concrete type \texttt{Vector3\{Int64\}}. FixedSizeArrays introduced
the dimensionality as a parameter as \texttt{Vector\{3,Int64\}}. This means the notion
of a fixed length vector can be abstracted over arbitrary dimensionality.

\cite{Pasko_Adzhiev_Comninos_2008}

\subsubsection{Numerical Robustness}

Numerical robustness is a perennial problem in computational geometry. Multiple
approaches exists for various numeric types. Floating points are by far
the most difficult to deal with. Tools such as Gappa have been developed so
algorithm writers can check their invariants when using floating points. \cite{Gappa}
Such tools complicate software development and are not an accessible option
for the casual researcher.

One of the most common problems formulated is to determine whether or not a
point is collinear with a line segment. Shewchuk has one of the most pragmatic
and robust treatments on this topic.\cite{Shewchuk} Kettner, et. al. have also
developed more examples where numerical robustness is critical. \cite{Kettner_Mehlhorn_Pion_Schirra_Yap_2008}

Julia's GeometricalPredicates package \footnote{\url{https://github.com/JuliaGeometry/GeometricalPredicates.jl}}
uses the approach outlined by Volker Springel, which requires all floating point
numbers to be scales between 1 and 2.\cite{Springel_2010} This has the downside
of significantly reducing the available resolution. \todo{why}

My preferred approach is to avoid the difficulties of floating point by working
within integer space. Developing a system around this is of interest. For
example, it should be possible to specify a minimum unit (e.g. microns)
and perform all computations in integer space assuming this does not exceed
the needed resolution. More importantly, modern CPUs have integrated 128 bit
Integer support. 170141183460469231731687303715884105727 is a lot of microns.

\subsubsection{Simplices}

Recently a Simplex type was added to GeometryTypes. A Simplex is defined
as the minimum convex set containing the specified points. The initial
prototype is very simple yet works well. Below we can see a 0th and 1st order
simplex constructed in $\mathbb{R}2$ and $\mathbb{R}3$
\begin{lstlisting}
julia> using GeometryTypes

julia> Simplex(Point(1,2))
GeometryTypes.Simplex{1,FixedSizeArrays.Point{2,Int64}}((FixedSizeArrays.Point{2,Int64}((1,2)),))

julia> Simplex(Point(1,2,3), Point(4,5,6))
GeometryTypes.Simplex{2,FixedSizeArrays.Point{3,Int64}}((FixedSizeArrays.Point{3,Int64}((1,2,3)),FixedSizeArrays.Point{3,Int64}((4,5,6))))
\end{lstlisting}

This representation makes it possible to write code based on the order and
dimensionality of a simplex. Few algorithms have been developed around the
new Simplex type, and unfortunately it is not integrated as a lower-level
construct for the other types yet.

In addition I would like to add a Simplical Complex type.\footnote{\url{https://en.wikipedia.org/wiki/Simplicial_complex}}

\subsubsection{Mesh Slicing}

Topology. Paper by Emmanuel would be a good start.

\subsection{Automatic Differentiation}

\subsubsection{Dual Numbers}
\begin{lstlisting}
julia> using DualNumbers

julia> f(x) = 2x+1
f (generic function with 1 method)

julia> f(Dual(1,1))
3 + 2du
\end{lstlisting}


\subsubsection{Rvachev Functions}


In the 1960's Vladimir Rvachev produced a method for handling the "inverse
problem of analytic geometry". His theory consists of functions which provide a
link between logical and set operations in geometric modeling and analytic
geometry.\cite{shapiro1991theory} I believe the following anecdote helps
elucidate the theory. While attempting to solve boundary value problems,
Rvachev formulated an equation of a square as
\begin{equation*}
a^2 + b^2 − x^2 − y^2 + \sqrt[]{( a^2 − x^2 )^2 +( b^2 − y^2 )^2} =0
\end{equation*}

Implicitly, the sides of a square can be defined as $x= +/- a$ and $y= +/- b$.
The union of these two is a square. By reducing the formulation of the square
we can generalize an expression for the union between two functions.
\begin{equation*}
\cup : f_1 + f_2 + \sqrt[]{f_1^2 +f_2^2} =0
\end{equation*}

Likewise we can see that intersections and negations can be formed for logical
completion.
\begin{equation*}
\cap : f_1 + f_2 - \sqrt[]{f_1^2 +f_2^2} =0 \\
\end{equation*}
\begin{equation*}
\neg : -f_1
\end{equation*}

These formulations can be modified for $C^m$ continuity for any $m$.
\todo{show this construction, it isn't obvious}
\cite{shapiro2007semi} In addition Pasko, et. al. have shown that Rvachev
functions can serve to replace a geometry kernel by creating logical
predicates. \cite{pasko1995function} Their research also establishes the
grounds for user interfaces and environment description. For this work a
practical implementation will most likely leverage their insights.
Rvachev and Shapiro have also shown that using the POLE-PLAST and SAGE
systems a user can generate complex semi-analytic geometry
as well.\cite{rvachev2000completeness} 

While a functional representation for geometry is mathematically enticing on
its own, the power it gives for numerical analysis might be its greatest
virtue. Numerical analysis justified the initial investigation by Rvachev
early on. A boundary value problem on a R-Function-predicate domain allows
for analysis without construction of a discrete mesh.\cite{rvachev2000completeness}

One of the most general expositions in the English language of R-Functions
applied to BVPs is
Vadim Shapiro's``Semi-Analytic Geometry with R-Functions". \cite{shapiro2007semi}
Unfortunately, no monographs about R-Functions exist in the English literature.
Most literature is in Russian, however many articles presenting applied
problems using the R-Function Method. \cite{voron2010}

Such a system for analytic geometry can be developed further. In the context
of an Eulerian flow field, a distance field over a function that
generates partial derivatives could be a fast numerical computation method.

\section{Conclusion}


\bibliography{references}

\end{document}

