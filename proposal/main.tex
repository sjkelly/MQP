\documentclass[a4paper]{article}

\usepackage[english]{babel}
\usepackage[utf8x]{inputenc}
\usepackage{amsmath}
\usepackage{graphicx}
\usepackage[square,sort,comma,numbers]{natbib}
\usepackage[colorinlistoftodos]{todonotes}
\bibliographystyle{ieeetr}  

\title{MQP Proposal}
\author{Stephen Kelly}

\begin{document}
\maketitle

\begin{abstract}
The rise of democratized and digital manufacturing has lead to a change in the demands of solid modeling tools. Rvachev functions appear to address some of the changing requirements by enabling a functional representation of boolean operations on solids. 
\end{abstract}

\section{Introduction}
\subsection{Social Impact and Principles of Application}
With the advent of consumer level 3D printing, there has been an increased demand for Computer Aided Design (CAD) software.  One potent aspect of 3D printing is the customization of designs by the end user. In order for this to be possible, models are designed to be parametric. Typically the designer will expose a few parameters for the user to adjust and thus customize to their liking. Many engineers use parametric design to adjust for manufacturing tolerances, and various component interfaces. However parametric design has a deeper possibility of allowing design to propagate further by adjusting to local material availability. One example of this is hardware sizes being adjustable from Metric to Imperial measure, in order to suit differing international markets. 

This idea is elucidated by Ronald Coase's economic idea of transaction cost between firms and people. \cite{CoaseNature} Transaction cost is the economic viability of further economic interaction. Naturally, distributed production could lead to an impact in economic and social interactions. Parametric design coupled with means of distributed production, such as 3D printing, helps to reduce the transaction cost of distributing ideas in the physical world. Designs then achieve their lowest transaction cost when parameters are tuned to allow adaption to non-printable components in different regions. 

In addition we might also seek to reduce the transaction cost between designers and contributors. Since ideas spread fastest when they are free and open, it is very important to harness all available improvements and alterations. Thus there are three groups we will primarily consider, originators, contributors, and users. The guiding principle from the application of this research will be the reduction of transaction costs between all three groups, in a manner conducive to interaction and sustainability.


\subsection{Solid Modeling}
Today, many people within the 3D printing community have to sacrifice usability and analysis tools to do parametric design. In addition, the dominant paradigm for solid modeling representation has remained unchanged since the 1970's, relying primarily on Boundary Representation (B-Rep). While B-Rep is intuitive for users of a graphical environment, it lacks as a textual representation. B-Rep relies primarily on the manipulation of edges, vertices, and faces to build a model. While this methods is natural for the engineer and designer, it is far from conducive for parametric design. Many users have turned to the methods of Constructive Solid Geometry (CSG) in which geometric primitives are manipulated, in order to provide a level of abstraction for parametric design. 

Given the commercial dominance and ease of use of B-Rep and the succinct and programmable representation of CSG, it would be in our favor to develop a representation that could balance each of these properties. Previous research (TextCAD) has shown the natural propensity of CSG to remain functional in structure. Most CSG engines take advantage of this fact, however for the most part design analysis tools do not. It seems that a functionally structured boundary representation would lend it self very naturally to the discretization and numerical solutions of boundary value problems. Thus the question addressed should be related to mathematical representation and analysis of such parametric constructions in order to facilitate numerical techniques in an efficient and concise manner.  


\section{Mathematical Theory}
\subsection{Survey of Mathematics in Modern Solid Modeling}
In order to understand the solution, one must first understand the problem. Mathematics and technology are said to "stand on the shoulders" of their forefathers. As such, dogmatism can easily be introduced when only the solutions or axioms are understood. In order to introduce a new or improved mathematical method of solid modeling, it would be virtuous to understand the present day modeling techniques.

There will be three main topics of study, primarily centered around the methods, representations, and analysis of geometric modeling.


\subsection{Rvachev Functions}
Rvachev functions at first glance seem to have ambiguous application in applied mathematics. Put succinctly, a Rvachev function is an extension of \cite{shapiro2007semi}

\cite{shapiro1991theory}

\cite{pasko1995function}

\section{Technologies}
\subsection{Julia}
From my previous research and experience while building TextCAD, the natural structure of programmed parametric design became more clear. The desire to construct TextCAD rose from the shortcomings of the OpenSCAD. There were three important things learned from the project which can hopefully be addressed through new technologies. The first important aspect was that CSG is functional in structure. Regardless of how the document is formatted, upon parsing the Syntax Tree of the object grows serially. The second observation showed the possibilities that dynamic typing and object oriented programming could have on the distribution of design. The ability to access various dimensions and pass parts as dependencies enabled construction that was verbose, yet understandable. Some of the verbosity can be attributed to the Python language. The third observation was that matricies and vectors should be first class citizens.

Importantly, all these observations address the expressiveness and structure of the language. A promising language, Julia, was released in 2012 that directly addresses these concerns. It was designed from the start to be a technical computing language, rather than a systems programming language.\cite{bezanson2012julia} In essence, it was designed to represent technical problems stemming from the physical world, rather than the digital world. Many languages, such as MATLAB and R follow a similar principle. Many systems such as Sage seek to bring libraries for technical computing together using a system, or general purpose, programming language. 


%\todo[inline, color=green!40]{This is an inline comment.}

\section{Conclusion}

The possibilities of Rvachev functions are very enticing to an applied mathematician. While they do have admitted flaws with some geometric representations, the possible technological applications that could spring from such a study would have great social impact. In particular, allowing designers to work with analysis tools could change the manner in which they work. 


\bibliography{references}


\end{document}